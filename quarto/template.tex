% Options for packages loaded elsewhere
\PassOptionsToPackage{unicode}{hyperref}
\PassOptionsToPackage{hyphens}{url}
%
\documentclass[
  ignorenonframetext,
  aspectratio=169,
]{beamer}
\usepackage{pgfpages}
\setbeamertemplate{caption}[numbered]
\setbeamertemplate{caption label separator}{: }
\setbeamercolor{caption name}{fg=normal text.fg}
\beamertemplatenavigationsymbolsempty
% Prevent slide breaks in the middle of a paragraph
\widowpenalties 1 10000
\raggedbottom
\setbeamertemplate{part page}{
  \centering
  \begin{beamercolorbox}[sep=16pt,center]{part title}
    \usebeamerfont{part title}\insertpart\par
  \end{beamercolorbox}
}
\setbeamertemplate{section page}{
  \centering
  \begin{beamercolorbox}[sep=12pt,center]{part title}
    \usebeamerfont{section title}\insertsection\par
  \end{beamercolorbox}
}
\setbeamertemplate{subsection page}{
  \centering
  \begin{beamercolorbox}[sep=8pt,center]{part title}
    \usebeamerfont{subsection title}\insertsubsection\par
  \end{beamercolorbox}
}
\AtBeginPart{
  \frame{\partpage}
}
\AtBeginSection{
  \ifbibliography
  \else
    \frame{\sectionpage}
  \fi
}
\AtBeginSubsection{
  \frame{\subsectionpage}
}

\usepackage{amsmath,amssymb}
\usepackage{iftex}
\ifPDFTeX
  \usepackage[T1]{fontenc}
  \usepackage[utf8]{inputenc}
  \usepackage{textcomp} % provide euro and other symbols
\else % if luatex or xetex
  \usepackage{unicode-math}
  \defaultfontfeatures{Scale=MatchLowercase}
  \defaultfontfeatures[\rmfamily]{Ligatures=TeX,Scale=1}
\fi
\usepackage{lmodern}
\usetheme[]{defaultstyles/inno\_style.scss}
\ifPDFTeX\else  
    % xetex/luatex font selection
\fi
% Use upquote if available, for straight quotes in verbatim environments
\IfFileExists{upquote.sty}{\usepackage{upquote}}{}
\IfFileExists{microtype.sty}{% use microtype if available
  \usepackage[]{microtype}
  \UseMicrotypeSet[protrusion]{basicmath} % disable protrusion for tt fonts
}{}
\makeatletter
\@ifundefined{KOMAClassName}{% if non-KOMA class
  \IfFileExists{parskip.sty}{%
    \usepackage{parskip}
  }{% else
    \setlength{\parindent}{0pt}
    \setlength{\parskip}{6pt plus 2pt minus 1pt}}
}{% if KOMA class
  \KOMAoptions{parskip=half}}
\makeatother
\usepackage{xcolor}
\newif\ifbibliography
\ifLuaTeX
  \usepackage{luacolor}
  \usepackage[soul]{lua-ul}
\else
  \usepackage{soul}
  \makeatletter
  \let\HL\hl
  \renewcommand\hl{% fix for beamer highlighting
    \let\set@color\beamerorig@set@color
    \let\reset@color\beamerorig@reset@color
    \HL}
  \makeatother
  
\fi
\setlength{\emergencystretch}{3em} % prevent overfull lines
\setcounter{secnumdepth}{-\maxdimen} % remove section numbering

\usepackage{color}
\usepackage{fancyvrb}
\newcommand{\VerbBar}{|}
\newcommand{\VERB}{\Verb[commandchars=\\\{\}]}
\DefineVerbatimEnvironment{Highlighting}{Verbatim}{commandchars=\\\{\}}
% Add ',fontsize=\small' for more characters per line
\newenvironment{Shaded}{}{}
\newcommand{\AlertTok}[1]{\textcolor[rgb]{0.16,0.16,0.16}{\textbf{\colorbox[rgb]{0.80,0.14,0.11}{#1}}}}
\newcommand{\AnnotationTok}[1]{\textcolor[rgb]{0.60,0.59,0.10}{#1}}
\newcommand{\AttributeTok}[1]{\textcolor[rgb]{0.84,0.60,0.13}{#1}}
\newcommand{\BaseNTok}[1]{\textcolor[rgb]{0.96,0.45,0.00}{#1}}
\newcommand{\BuiltInTok}[1]{\textcolor[rgb]{0.84,0.36,0.05}{#1}}
\newcommand{\CharTok}[1]{\textcolor[rgb]{0.69,0.38,0.53}{#1}}
\newcommand{\CommentTok}[1]{\textcolor[rgb]{0.57,0.51,0.45}{#1}}
\newcommand{\CommentVarTok}[1]{\textcolor[rgb]{0.57,0.51,0.45}{#1}}
\newcommand{\ConstantTok}[1]{\textcolor[rgb]{0.69,0.38,0.53}{\textbf{#1}}}
\newcommand{\ControlFlowTok}[1]{\textcolor[rgb]{0.80,0.14,0.11}{\textbf{#1}}}
\newcommand{\DataTypeTok}[1]{\textcolor[rgb]{0.84,0.60,0.13}{#1}}
\newcommand{\DecValTok}[1]{\textcolor[rgb]{0.96,0.45,0.00}{#1}}
\newcommand{\DocumentationTok}[1]{\textcolor[rgb]{0.60,0.59,0.10}{#1}}
\newcommand{\ErrorTok}[1]{\textcolor[rgb]{0.80,0.14,0.11}{\underline{#1}}}
\newcommand{\ExtensionTok}[1]{\textcolor[rgb]{0.41,0.62,0.42}{\textbf{#1}}}
\newcommand{\FloatTok}[1]{\textcolor[rgb]{0.96,0.45,0.00}{#1}}
\newcommand{\FunctionTok}[1]{\textcolor[rgb]{0.41,0.62,0.42}{#1}}
\newcommand{\ImportTok}[1]{\textcolor[rgb]{0.41,0.62,0.42}{#1}}
\newcommand{\InformationTok}[1]{\textcolor[rgb]{0.16,0.16,0.16}{\colorbox[rgb]{0.51,0.65,0.60}{#1}}}
\newcommand{\KeywordTok}[1]{\textcolor[rgb]{0.24,0.22,0.21}{\textbf{#1}}}
\newcommand{\NormalTok}[1]{\textcolor[rgb]{0.24,0.22,0.21}{#1}}
\newcommand{\OperatorTok}[1]{\textcolor[rgb]{0.24,0.22,0.21}{#1}}
\newcommand{\OtherTok}[1]{\textcolor[rgb]{0.41,0.62,0.42}{#1}}
\newcommand{\PreprocessorTok}[1]{\textcolor[rgb]{0.84,0.36,0.05}{#1}}
\newcommand{\RegionMarkerTok}[1]{\textcolor[rgb]{0.57,0.51,0.45}{\colorbox[rgb]{0.98,0.96,0.84}{#1}}}
\newcommand{\SpecialCharTok}[1]{\textcolor[rgb]{0.69,0.38,0.53}{#1}}
\newcommand{\SpecialStringTok}[1]{\textcolor[rgb]{0.60,0.59,0.10}{#1}}
\newcommand{\StringTok}[1]{\textcolor[rgb]{0.60,0.59,0.10}{#1}}
\newcommand{\VariableTok}[1]{\textcolor[rgb]{0.27,0.52,0.53}{#1}}
\newcommand{\VerbatimStringTok}[1]{\textcolor[rgb]{0.60,0.59,0.10}{#1}}
\newcommand{\WarningTok}[1]{\textcolor[rgb]{0.16,0.16,0.16}{\colorbox[rgb]{0.98,0.74,0.18}{#1}}}

\providecommand{\tightlist}{%
  \setlength{\itemsep}{0pt}\setlength{\parskip}{0pt}}\usepackage{longtable,booktabs,array}
\usepackage{calc} % for calculating minipage widths
\usepackage{caption}
% Make caption package work with longtable
\makeatletter
\def\fnum@table{\tablename~\thetable}
\makeatother
\usepackage{graphicx}
\makeatletter
\def\maxwidth{\ifdim\Gin@nat@width>\linewidth\linewidth\else\Gin@nat@width\fi}
\def\maxheight{\ifdim\Gin@nat@height>\textheight\textheight\else\Gin@nat@height\fi}
\makeatother
% Scale images if necessary, so that they will not overflow the page
% margins by default, and it is still possible to overwrite the defaults
% using explicit options in \includegraphics[width, height, ...]{}
\setkeys{Gin}{width=\maxwidth,height=\maxheight,keepaspectratio}
% Set default figure placement to htbp
\makeatletter
\def\fps@figure{htbp}
\makeatother
% definitions for citeproc citations
\NewDocumentCommand\citeproctext{}{}
\NewDocumentCommand\citeproc{mm}{%
  \begingroup\def\citeproctext{#2}\cite{#1}\endgroup}
\makeatletter
 % allow citations to break across lines
 \let\@cite@ofmt\@firstofone
 % avoid brackets around text for \cite:
 \def\@biblabel#1{}
 \def\@cite#1#2{{#1\if@tempswa , #2\fi}}
\makeatother
\newlength{\cslhangindent}
\setlength{\cslhangindent}{1.5em}
\newlength{\csllabelwidth}
\setlength{\csllabelwidth}{3em}
\newenvironment{CSLReferences}[2] % #1 hanging-indent, #2 entry-spacing
 {\begin{list}{}{%
  \setlength{\itemindent}{0pt}
  \setlength{\leftmargin}{0pt}
  \setlength{\parsep}{0pt}
  % turn on hanging indent if param 1 is 1
  \ifodd #1
   \setlength{\leftmargin}{\cslhangindent}
   \setlength{\itemindent}{-1\cslhangindent}
  \fi
  % set entry spacing
  \setlength{\itemsep}{#2\baselineskip}}}
 {\end{list}}
\usepackage{calc}
\newcommand{\CSLBlock}[1]{\hfill\break\parbox[t]{\linewidth}{\strut\ignorespaces#1\strut}}
\newcommand{\CSLLeftMargin}[1]{\parbox[t]{\csllabelwidth}{\strut#1\strut}}
\newcommand{\CSLRightInline}[1]{\parbox[t]{\linewidth - \csllabelwidth}{\strut#1\strut}}
\newcommand{\CSLIndent}[1]{\hspace{\cslhangindent}#1}

\makeatletter
\@ifpackageloaded{caption}{}{\usepackage{caption}}
\AtBeginDocument{%
\ifdefined\contentsname
  \renewcommand*\contentsname{Table of contents}
\else
  \newcommand\contentsname{Table of contents}
\fi
\ifdefined\listfigurename
  \renewcommand*\listfigurename{List of Figures}
\else
  \newcommand\listfigurename{List of Figures}
\fi
\ifdefined\listtablename
  \renewcommand*\listtablename{List of Tables}
\else
  \newcommand\listtablename{List of Tables}
\fi
\ifdefined\figurename
  \renewcommand*\figurename{Figure}
\else
  \newcommand\figurename{Figure}
\fi
\ifdefined\tablename
  \renewcommand*\tablename{Table}
\else
  \newcommand\tablename{Table}
\fi
}
\@ifpackageloaded{float}{}{\usepackage{float}}
\floatstyle{ruled}
\@ifundefined{c@chapter}{\newfloat{codelisting}{h}{lop}}{\newfloat{codelisting}{h}{lop}[chapter]}
\floatname{codelisting}{Listing}
\newcommand*\listoflistings{\listof{codelisting}{List of Listings}}
\usepackage{amsthm}
\theoremstyle{plain}
\newtheorem{theorem}{Theorem}[section]
\theoremstyle{remark}
\AtBeginDocument{\renewcommand*{\proofname}{Proof}}
\newtheorem*{remark}{Remark}
\newtheorem*{solution}{Solution}
\newtheorem{refremark}{Remark}[section]
\newtheorem{refsolution}{Solution}[section]
\makeatother
\makeatletter
\makeatother
\makeatletter
\@ifpackageloaded{caption}{}{\usepackage{caption}}
\@ifpackageloaded{subcaption}{}{\usepackage{subcaption}}
\makeatother
\makeatletter
\@ifpackageloaded{tcolorbox}{}{\usepackage[skins,breakable]{tcolorbox}}
\makeatother
\makeatletter
\@ifundefined{shadecolor}{\definecolor{shadecolor}{rgb}{.97, .97, .97}}{}
\makeatother
\makeatletter
\makeatother
\makeatletter
\ifdefined\Shaded\renewenvironment{Shaded}{\begin{tcolorbox}[breakable, boxrule=0pt, borderline west={3pt}{0pt}{shadecolor}, sharp corners, frame hidden, interior hidden, enhanced]}{\end{tcolorbox}}\fi
\makeatother
\ifLuaTeX
  \usepackage{selnolig}  % disable illegal ligatures
\fi
\usepackage{bookmark}

\IfFileExists{xurl.sty}{\usepackage{xurl}}{} % add URL line breaks if available
\urlstyle{same} % disable monospaced font for URLs
\hypersetup{
  pdftitle={The Titile of Fancy Presentation},
  pdfauthor={Simeon Nedelchev},
  hidelinks,
  pdfcreator={LaTeX via pandoc}}

\title{The Titile of Fancy Presentation}
\subtitle{The cool looking html slides}
\author{Simeon Nedelchev}
\date{Jun 8, 2024}
\logo{\includegraphics{images/logos/inno.png}}

\begin{document}
\frame{\titlepage}

\begin{frame}{Control Elements}
\phantomsection\label{control-elements}
\begin{figure}

\begin{minipage}{0.05\linewidth}
\includegraphics[width=0.42708in,height=\textheight]{images/icons/menu.png}\end{minipage}%
%
\begin{minipage}{0.95\linewidth}
Toggle the slide menu with the menu button (top left) to go to other
slides and access presentation tools.\end{minipage}%

\end{figure}%

\begin{figure}

\begin{minipage}{0.05\linewidth}
\includegraphics[width=0.42708in,height=\textheight]{images/icons/chalkboard.png}\end{minipage}%
%
\begin{minipage}{0.95\linewidth}
Use the chalkboard button (bottom left) of the slide to toggle the
chalkboard.\end{minipage}%

\end{figure}%

\begin{figure}

\begin{minipage}{0.05\linewidth}
\includegraphics[width=0.42708in,height=\textheight]{images/icons/canvas.png}\end{minipage}%
%
\begin{minipage}{0.95\linewidth}
Use the notes canvas button at the (bottom left) to toggle drawing on
top of the current slide.\end{minipage}%

\end{figure}%
\end{frame}

\begin{frame}[fragile]{Hotkeys}
\phantomsection\label{hotkeys}
There are some features and hot keys:

\begin{columns}[T]
\begin{column}{0.5\textwidth}
\begin{longtable}[]{@{}ll@{}}
\toprule\noalign{}
\endhead
→ , ↓ , ← , ↑ & Navigation \\
Alt + ←/↑/→/↓ & Navigate without fragments \\
Shift + ←/↑/→/↓ & Jump to first/last slide \\
CTRL + Shift + F & Search \\
ESC, O & Slide overview \\
\bottomrule\noalign{}
\end{longtable}
\end{column}

\begin{column}{0.5\textwidth}
\begin{longtable}[]{@{}ll@{}}
\toprule\noalign{}
\endhead
F & Fullscreen \\
S & Speaker notes view \\
B, C & Toggle chalkboard/notes canvas \\
DEL, BACKSPACE & Clear/reset drawings on slide \\
M & Toggle menu \\
\bottomrule\noalign{}
\end{longtable}
\end{column}
\end{columns}

The full list of hotkeys is accesable via key ``\texttt{?}''
\end{frame}

\section{Text, Layout, Fragments}\label{text-layout-fragments}

\begin{frame}[fragile]{Text Formatting}
\phantomsection\label{text-formatting}
The text is formatted via markdown or html, \textbf{bold},
\emph{italic}, underline, superscript\textsuperscript{2},
subscript\textsubscript{2}, \st{strikethrough}, \texttt{verbatim\ code},
\url{https://google.org}, \href{https://google.com}{link to google}

\begin{block}{Lower Header}
\phantomsection\label{lower-header}
Here is a footnote reference\footnote<.->{Here is the footnote.}

You may create the footer like this
\end{block}
\end{frame}

\begin{frame}{Lists, Layout, Tabsets}
\phantomsection\label{lists-layout-tabsets}
The layout can be changed, for instance as 2 columns:

\begin{columns}[T]
\begin{column}{0.5\textwidth}
Unordered list:

\begin{itemize}
\tightlist
\item
  unordered list 1

  \begin{itemize}
  \tightlist
  \item
    sub-item 1
  \end{itemize}
\item
  unordered list 2
\end{itemize}
\end{column}

\begin{column}{0.5\textwidth}
Ordered list:

\begin{enumerate}
\tightlist
\item
  unordered list 1

  \begin{enumerate}
  [i)]
  \tightlist
  \item
    sub-item 1
  \end{enumerate}
\item
  item 2
\end{enumerate}
\end{column}
\end{columns}

\begin{block}{First Tab}
Some content is in the first tab
\end{block}

\begin{block}{Second Tab}
while some in the second\ldots{}
\end{block}

\begin{block}{Third Tab}
Hello there
\end{block}
\end{frame}

\begin{frame}{Fragments and Pause}
\phantomsection\label{fragments-and-pause}
Incremental text display and animation with fragments:

\begin{figure}

\begin{minipage}{0.50\linewidth}
Strike\end{minipage}%
%
\begin{minipage}{0.50\linewidth}
Color highlight\end{minipage}%
\newline
\begin{minipage}{0.50\linewidth}
Fade in\end{minipage}%
%
\begin{minipage}{0.50\linewidth}
Slide up while fading in\end{minipage}%
\newline
\begin{minipage}{0.50\linewidth}
Slide left while fading in\end{minipage}%
%
\begin{minipage}{0.50\linewidth}
Fade in then semi out\end{minipage}%

\end{figure}%

\pause

one can also ``pause'' the slide\ldots.

\pause

and continue with something
\end{frame}

\section{Code, Tables, Figures,
Videos}\label{code-tables-figures-videos}

\begin{frame}[fragile]{Codeu}
\phantomsection\label{codeu}
The code blocks are orginized as in Markdown:

\begin{Shaded}
\begin{Highlighting}[]
\ImportTok{import}\NormalTok{ numpy }\ImportTok{as}\NormalTok{ np}

\NormalTok{r }\OperatorTok{=}\NormalTok{ np.arange(}\DecValTok{0}\NormalTok{, }\DecValTok{2}\NormalTok{, }\FloatTok{0.01}\NormalTok{)}
\NormalTok{theta }\OperatorTok{=} \DecValTok{2} \OperatorTok{*}\NormalTok{ np.pi }\OperatorTok{*}\NormalTok{ r}
\end{Highlighting}
\end{Shaded}

The lines can be highlighted via argument \texttt{code-line-numbers}:

\begin{Shaded}
\begin{Highlighting}[numbers=left,,]
\ImportTok{import}\NormalTok{ numpy }\ImportTok{as}\NormalTok{ np}
\ImportTok{import}\NormalTok{ matplotlib.pyplot }\ImportTok{as}\NormalTok{ plt}

\NormalTok{r }\OperatorTok{=}\NormalTok{ np.arange(}\DecValTok{0}\NormalTok{, }\DecValTok{2}\NormalTok{, }\FloatTok{0.01}\NormalTok{)}
\NormalTok{theta }\OperatorTok{=} \DecValTok{2} \OperatorTok{*}\NormalTok{ np.pi }\OperatorTok{*}\NormalTok{ r}
\NormalTok{fig, ax }\OperatorTok{=}\NormalTok{ plt.subplots(subplot\_kw}\OperatorTok{=}\NormalTok{\{}\StringTok{\textquotesingle{}projection\textquotesingle{}}\NormalTok{: }\StringTok{\textquotesingle{}polar\textquotesingle{}}\NormalTok{\})}
\NormalTok{ax.plot(theta, r)}
\NormalTok{ax.set\_rticks([}\FloatTok{0.5}\NormalTok{, }\DecValTok{1}\NormalTok{, }\FloatTok{1.5}\NormalTok{, }\DecValTok{2}\NormalTok{])}
\NormalTok{ax.grid(}\VariableTok{True}\NormalTok{)}
\NormalTok{plt.show()}
\end{Highlighting}
\end{Shaded}
\end{frame}

\begin{frame}{Tables}
\phantomsection\label{tables}
Tables can be orginized via markdown, while providing control over
column-widths and text location

\begin{columns}[T]
\begin{column}{0.15\textwidth}
\end{column}

\begin{column}{0.3\textwidth}
\begin{longtable}[]{@{}
  >{\raggedright\arraybackslash}p{(\columnwidth - 2\tabcolsep) * \real{0.7500}}
  >{\raggedright\arraybackslash}p{(\columnwidth - 2\tabcolsep) * \real{0.2500}}@{}}
\toprule\noalign{}
\begin{minipage}[b]{\linewidth}\raggedright
fruit
\end{minipage} & \begin{minipage}[b]{\linewidth}\raggedright
price
\end{minipage} \\
\midrule\noalign{}
\endhead
apple & 2.05 \\
pear & 1.37 \\
orange & 3.09 \\
\bottomrule\noalign{}
\end{longtable}
\end{column}

\begin{column}{0.5\textwidth}
\begin{longtable}[]{@{}
  >{\raggedleft\arraybackslash}p{(\columnwidth - 4\tabcolsep) * \real{0.1389}}
  >{\raggedright\arraybackslash}p{(\columnwidth - 4\tabcolsep) * \real{0.1250}}
  >{\centering\arraybackslash}p{(\columnwidth - 4\tabcolsep) * \real{0.2639}}@{}}
\toprule\noalign{}
\begin{minipage}[b]{\linewidth}\raggedleft
Right
\end{minipage} & \begin{minipage}[b]{\linewidth}\raggedright
Left
\end{minipage} & \begin{minipage}[b]{\linewidth}\centering
Centered
\end{minipage} \\
\midrule\noalign{}
\endhead
Bananas & \$1.34 & built-in wrapper \\
Oranges & \$2.13 & some text \\
\bottomrule\noalign{}
\end{longtable}
\end{column}
\end{columns}

Tables can be cross referenced as Table~\ref{tbl-first}

\begin{longtable}[]{@{}llllll@{}}
\toprule\noalign{}
Col1 & Col2 & Col3 & Col4 & Col5 & Col6 \\
\midrule\noalign{}
\endfirsthead
\toprule\noalign{}
Col1 & Col2 & Col3 & Col4 & Col5 & Col6 \\
\midrule\noalign{}
\endhead
A & B & C & D & E & F \\
E & F & G & H & I & J \\
A & G & G & K & L & M \\
\bottomrule\noalign{}
\caption{First Table}\label{tbl-first}\tabularnewline
\end{longtable}
\end{frame}

\begin{frame}{Figures}
\phantomsection\label{figures}
\begin{figure}

\centering{

\includegraphics{images/figures/inno_university.jpg}

}

\caption{\label{fig-inno}Innopolis University}

\end{figure}%

One may refer to figure Figure~\ref{fig-inno}
\end{frame}

\begin{frame}{Subfigures}
\phantomsection\label{subfigures}
\begin{figure}

\begin{minipage}{0.50\linewidth}

\begin{figure}[H]

\centering{

\includegraphics{images/figures/inno_gym.jpg}

}

\caption{\label{fig-inno_gym}Outside Gym}

\end{figure}%

\end{minipage}%
%
\begin{minipage}{0.50\linewidth}

\begin{figure}[H]

\centering{

\includegraphics{images/figures/inno_pool.jpg}

}

\caption{\label{fig-inno_pool}Inside Gym}

\end{figure}%

\end{minipage}%

\end{figure}%
\end{frame}

\begin{frame}{Videos}
\phantomsection\label{videos}
\url{https://www.youtube.com/watch?v=ZnWfmkMXbyY}
\end{frame}

\section{Equations, Theorems, Citations,
Bibliography}\label{equations-theorems-citations-bibliography}

\begin{frame}{LaTeX Equations}
\phantomsection\label{latex-equations}
\href{https://www.mathjax.org/}{MathJax} rendering of equations to HTML.
Inline equations are included within a line of text \(E=mc^2\).

Displayed equations set apart from the text: \[
\int_{-\infty}^\infty e^{-x^2}dx = \sqrt{\pi}
\]

Equations can be auotumathecally numbered:

\begin{equation}\phantomsection\label{eq-nearest_orth}{
\mathbf {O} ={\underset {\Omega }{\operatorname {argmin} }}\|\mathbf {A} {\boldsymbol {\Omega }}-\mathbf {B} \|_{F}\quad {\text{subject to}}\quad {\boldsymbol {\Omega }}^{\textsf {T}}{\boldsymbol {\Omega }}=\mathbf {I}
}\end{equation}

and referenced as (Equation~\ref{eq-nearest_orth})
\end{frame}

\begin{frame}[fragile]{Theorems, Lemmas, Exercises}
\phantomsection\label{theorems-lemmas-exercises}
\begin{theorem}[Fundamental theorem of
calculus]\protect\hypertarget{thm-calculus}{}\label{thm-calculus}

Let \(f\) be a continuous real-valued function defined on a closed
interval \([a, b]\). Let \(F\) be the function defined, for all \(x\) in
\([a, b]\), by \(F(x)=\int _{a}^{x}f(t)\,dt.\) Then \(F\) is uniformly
continuous on \([a, b]\) and differentiable on the open interval
\((a, b)\), and \(F'(x)=f(x)\) for all \(x\) in \((a, b)\) so \(F\) is
an antiderivative of \(f\)

\end{theorem}

Theorems are cross-referable as Theorem~\ref{thm-calculus}.

There are a number of theorem variations supported, each with their own
label prefix: lemmas are \texttt{\#lem}, proposition are \texttt{\#prp}
etc,
\href{https://quarto.org/docs/authoring/cross-references.html\#theorems-and-proofs}{read
more}
\end{frame}

\begin{frame}[fragile]{Citations}
\phantomsection\label{citations}
The \texttt{bib} references and \texttt{csl} is fully supported, one may
cite the articles, books etc.

According to Newton's Philosophiæ Naturalis Principia Mathematica
{[}1{]}.

Dantzig's work on linear programming {[}2{]}.

Feynman's space-time approach to non-relativistic quantum mechanics
{[}3{]}.

Celebrated Turing's work on computable numbers {[}4{]}.
\end{frame}

\begin{frame}{Bibliography}
\phantomsection\label{bibliography}
\phantomsection\label{refs}
\begin{CSLReferences}{0}{0}
\bibitem[\citeproctext]{ref-newton1687principia}
\CSLLeftMargin{{[}1{]} }%
\CSLRightInline{I. Newton, \emph{Philosophiæ naturalis principia
mathematica}. London: J. Streater, 1687.}

\bibitem[\citeproctext]{ref-dantzig1947maximization}
\CSLLeftMargin{{[}2{]} }%
\CSLRightInline{G. B. Dantzig, \emph{Maximization of a linear function
subject to linear inequalities}. New York: Wiley, 1947.}

\bibitem[\citeproctext]{ref-feynman1948space}
\CSLLeftMargin{{[}3{]} }%
\CSLRightInline{R. P. Feynman, {``Space-time approach to
non-relativistic quantum mechanics,''} \emph{Reviews of Modern Physics},
vol. 20, no. 2, pp. 367--387, 1948.}

\bibitem[\citeproctext]{ref-turing1937computable}
\CSLLeftMargin{{[}4{]} }%
\CSLRightInline{A. M. Turing, {``On computable numbers, with an
application to the entscheidungsproblem,''} \emph{Proceedings of the
London Mathematical Society}, vol. 2, no. 42, pp. 230--265, 1937.}

\end{CSLReferences}
\end{frame}

\begin{frame}{And More\ldots{}}
\phantomsection\label{and-more}
\begin{itemize}
\tightlist
\item
  \href{https://quarto.org/docs/presentations/revealjs/advanced.html\#touch-navigation}{Touch}
  - presentations look great on mobile, swipe to navigate slides.
\item
  \href{https://quarto.org/docs/presentations/revealjs/presenting.html\#multiplex}{Multiplex}
  - allows your audience to follow the slides of the presentation you
  are controlling on their own phone, tablet or laptop.
\item
  \href{https://quarto.org/docs/presentations/revealjs/themes.html}{Themes}
  - 10 Built-in or
  \href{https://quarto.org/docs/presentations/revealjs/themes.html\#creating-themes}{create
  your own}
\item
  \href{https://quarto.org/docs/presentations/revealjs/presenting.html\#auto-slide}{Auto-Slide}
  - step through slides automatically, without any user input
\item
  \href{https://quarto.org/docs/interactive/widgets/jupyter.html}{Widgets}
  - include Jupyter widgets and htmlwidgets in your presentations
\item
  \href{https://quarto.org/docs/authoring/notebook-embed.html}{Embedding
  Jupyter Notebooks} - include the output of an external Jupyter
  notebook in a Quarto document.
\end{itemize}

Follow \url{https://quarto.org/} for more!
\end{frame}



\end{document}
