\begin{abstract}

    Управление телеуправляемым необитаемым подводным аппаратом (ТНПА) в подводной среде является сложной задачей из-за их нелинейности и непредсказуемости водной среды. В данной работе исследуется усовершенствование управления в скользящем режиме (SMC) для ТНПА с помощью выпуклой оптимизации для улучшения стабильности и производительности при уменьшении эффекта дребезга. Разработана комплексная математическая модель ТНПА, учитывающая динамические и кинематические свойства, а также неопределенность окружающей среды. Разработана оптимизированная схема SMC, включающая выпуклую оптимизацию для динамической настройки параметров управления, что позволяет минимизировать дребезжание и повысить устойчивость. Проверка с помощью моделирования демонстрирует значительные улучшения в отслеживании траектории и отклонении возмущений. Усовершенствованная схема SMC доказала свою эффективность при моделировании, что подтверждает ее практическую применимость. Данное исследование способствует развитию подводной робототехники, предоставляя надежную стратегию управления, которая повышает стабильность и эффективность ТНПА, обеспечивая выполнение сложных подводных задач, таких как инспекция трубопроводов и мониторинг окружающей среды.

\end{abstract}