\chapter{Заключение}
\label{chap:conclusion}

В данной диссертации рассматривается проблема управления телеуправляемыми необитаемыми подводными аппаратами (ТНПА) в подводной среде с нелинейностью и неопределенностью. Она посвящена улучшению управления в скользящем режиме (SMC) с использованием методов выпуклой оптимизации для повышения стабильности, производительности и снижения эффекта дребезга.

Основной вклад в исследование заключается в разработке комплексной математической модели ТНПА и создании усовершенствованной схемы SMC с использованием выпуклой оптимизации. Предложенная схема управления была проверена с помощью моделирования и экспериментов, показав улучшенное отслеживание траектории и отклонение возмущений по сравнению с традиционными методами.

Полученные результаты свидетельствуют об улучшении стабильности и производительности усовершенствованной схемы SMC, обеспечивающей более высокую надежность и эффективность работы ТНПА. Уменьшение дребезга было достигнуто с помощью выпуклой оптимизации, что повысило эффективность управления и снизило износ компонентов ТНПА.

Области для дальнейших исследований включают повышение вычислительной эффективности процесса выпуклой оптимизации и тестирование предложенной стратегии управления в различных подводных условиях. Кроме того, в будущем можно будет изучить возможности интеграции передовых сенсорных технологий для повышения адаптивности и интеллектуальности систем управления ТНПА.

Таким образом, данная диссертация вносит значительный вклад в область подводной робототехники, предлагая надежную и эффективную стратегию управления для ТНПА. Интеграция выпуклой оптимизации с управлением в скользящем режиме предлагает решение, которое обеспечивает баланс между производительностью, стабильностью и надежностью, поддерживая разработку более способных и надежных подводных роботизированных систем в сложных условиях.


