\chapter{Обзор литературы}
\label{chap:lr}

\chaptermark{Second Chapter Heading}

\section{Физические характеристики подводных аппаратов}

    Согласно \cite{rov_review}, механическая структура ТНПА включает камеру наблюдения, датчики для сбора навигационных данных и актуаторы для управления направлением. Сравнительное исследование \cite{overview2} показало, что на работу ТНПА влияют точность сенсорных систем и конструкция движителей. Непредсказуемость подводных течений, сопротивления и динамики плавучести также влияет на работу ТНПА, усложняя моделирование.

    Исследования \cite{rov_application, overview} показали важную роль ТНПА в промышленных приложениях, морской разведке нефти и газа, патрулировании и наблюдении. Следовательно, система управления должна отслеживать положение и удерживать станцию в условиях неопределенности параметров и окружающей среды.

\section{Математическое моделирование}

    Тор И. Фоссен \cite{fossen:guidance} описал основы математического моделирования морских аппаратов. ТНПА моделировался как единое жесткое тело (SRB), упрощая моделирование и отражая динамику системы.

    Кинематически ТНПА имеет шесть степеней свободы (DoFs). Ориентация, выраженная в углах поворота, может привести к сингулярности, что решается кватернионным представлением \cite{quat_smc}. Динамика описывается Вторым законом Ньютона и уравнением Эйлера-Лагранжа, формируя набор нелинейных уравнений.

    Некоторые аспекты динамики ТНПА требуют эмпирических оценок из-за их сложности \cite{fossen:guidance, bluerov}. При движении через вязкую среду инерция окружающей жидкости и демпфирование воды играют важную роль. Также принимаются во внимание силы плавучести и гравитации. В исследовании \cite{bluerov} говорится, что создание точной модели подруливающего устройства сложно из-за различных факторов, таких как модели двигателей и гидродинамические эффекты.

    При таких упрощениях система управления не может самостоятельно обеспечить эффективное управление такой неопределенной динамикой. В результате для точного отслеживания положения ТНПА необходима надежная система управления.

\section{Система управления}

    Управление ТНПА включает стабилизацию аппарата и выполнение инструкций оператора. Система управления должна справляться с возмущениями, вызванными изменениями параметров и окружающей среды \cite{overview, control_overview}. Основные проблемы управления ТНПА:
    \begin{enumerate}
        \item Немоделируемые элементы, такие как дополнительная масса и гидродинамические коэффициенты.
        \item Сильно нелинейная динамика подводной среды.
    \end{enumerate}

    Подход обратной динамики \cite{spong_book} прост, но ограничен зависимостью от точных параметров модели и неспособностью справляться с возмущениями. Классическим подходом является управление в скользящем режиме (SMC), которое было представлено В. И. Уткиным \cite{utkin}. Однако стандартное SMC вводит высокочастотные сигналы, которые могут вызвать переключение актуатора и, следовательно, сократить срок его службы \cite{slotine, spong}. Выпуклая оптимизация позволяет достичь баланса между эффективностью управления и долговечностью привода, повышая надежность и устойчивость к возмущениям \cite{utkin_opt, utkin_book}.

\section{Заключение}

    Обзор литературы рассмотрел общие характеристики, математическое моделирование и решения по управлению подводными аппаратами. Обзор выявил пробел в надежном управлении ТНПА, особенно в условиях неопределенности параметров и окружающей среды. Первоначальная схема управления в скользящем режиме обеспечивала устойчивость, но страдала от высокочастотных колебаний. Решение предложено в комбинации выпуклой оптимизации и скользящего режима.