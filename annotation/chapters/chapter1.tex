\chapter{Введение}
\label{chap:intro}

Роботизированные системы, в частности подводные аппараты, такие как телеуправляемые необитаемые подводные аппараты (ТНПА), привлекли значительное внимание в последние годы благодаря их широкому применению в промышленной, научной и военной сферах. ТНПА используются для различных целей, включая подводную разведку, инспекцию трубопроводов и мониторинг окружающей среды. Однако управление этими аппаратами является сложной задачей из-за нелинейной и непредсказуемой природы подводной среды. Поэтому эффективные стратегии управления необходимы для обеспечения стабильности и эффективности ТНПА в таких условиях.

Стратегии управления ТНПА эволюционировали, при этом надежные методы управления, такие как управление в скользящем режиме (SMC), получили широкое распространение благодаря своей эффективности в работе с нелинейностью и неопределенностью системы. Несмотря на эти достижения, существующие методы управления для ТНПА сталкиваются с рядом проблемами. Традиционный метод скользящего режима, хотя и является надежным, часто приводит к явлению известному как «дребезг», который может привести к износу механических компонентов и снижает общую эффективность системы управления. Кроме того, многие алгоритмы управления с трудом адаптируются к изменениям подводной среды в реальном времени, таким как переменная сила течения и непредвиденные препятствия. Эти ограничения подчеркивают необходимость в улучшенных стратегиях управления, которые могут обеспечить как устойчивость, так и адаптивность.

Основной целью данного исследования является улучшение управления телеуправляемыми необитаемыми подводными аппаратами (ТНПА) путем разработки улучшенной схемы управления в скользящем режиме (SMC), использующей методы выпуклой оптимизации. Данная дипломная работа направлена на решение существующих проблем дребезга и адаптивности путем динамической настройки параметров управления с помощью выпуклой оптимизации.
