\chapter{Результаты}
\label{chap:impl}

В этой главе мы оцениваем эффективность различных стратегий управления для BlueROV Heavy, уделяя особое внимание отслеживанию положения, обработке возмущений, а также сравнивая точность и энергопотребление различных методов.

\textbf{Отслеживание положения}: имеет решающее значение для выполнения подводных задач, поскольку оно определяет, насколько хорошо ТНПА следует заданной траектории:
\begin{itemize}
    \item Обратная динамика (ID): Проявляет значительные ошибки в направлении $z$ и колебания стабилизации.
    \item Скользящий режим (SM): Демонстрирует устойчивое слежение и лучше справляется с нелинейностью и неопределенностью.
    \item Оптимизированный скользящий режим (QP): Повышает точность и эффективность за счет оптимизации управляющих воздействий.
\end{itemize}

\begin{figure}[h]
    \centering
    \includegraphics*[width=0.99\textwidth]{id_pos2}
    \includegraphics*[width=0.99\textwidth]{sm_pos2}
    \includegraphics*[width=0.99\textwidth]{qp_pos2}
    \caption{Эффективность отслеживания положения BlueROV Heavy. Сверху вниз: обратная динамика (a), скользящий режим (b) и оптимизированный скользящий режим (c).}
    \label{image:pos_tracking}
\end{figure}

\textbf{Влияние возмущений}: Подводная среда характеризуется такими возмущениями, как водные течения. В таких условиях системы управления работают следующим образом:
\begin{itemize}
    \item Обратная динамика (ID): Сохраняет стабильность, но имеет большие отклонения при возмущениях.
    \item Скользящий режим (SM): Демонстрирует повышенную устойчивость, снижая влияние возмущений.
    \item Оптимизированный скользящий режим (QP): Эффективно смягчает возмущения, сохраняя точность траектории.
\end{itemize}

\textbf{Сравнение методов управления}. Методы управления сравниваются по точности и энергопотреблению:
\begin{itemize}
    \item Обратная динамика (ID): Хорошая точность при идеальных условиях, но чувствительна к помехам и неточностям. Более высокое потребление энергии из-за зависимости от точных моделей.
    \item Скользящий режим (SM): Повышенная точность при неопределенности и возмущениях. Повышение эффективности за счет адаптации к изменяющимся условиям.
    \item Оптимизированный скользящий режим (QP): Наивысшая точность благодаря оптимизированным входам управления и надежной обработке возмущений. Наилучшая энергоэффективность за счет оптимизации использования подруливающих устройств.
\end{itemize}

    \begin{figure}[h]
        \begin{center}
            \includegraphics*[width=0.49\textwidth]{e_norm}
            \includegraphics*[width=0.49\textwidth]{u_norm}
        \end{center}
        \caption{Ошибка управления и показатели энергоэффективности систем управления: обратная динамика (id), скользящий режим (sm) и оптимизированный скользящий режим (qp)}
    \end{figure}

Оптимизированная система управления со скользящим режимом (QP) оказалась лучшей в целом, обеспечив эффективный баланс между точностью, надежностью и энергоэффективностью.

