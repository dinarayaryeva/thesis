\chapter{Моделирование и управление подводными аппаратами}
\label{chap:mat}

\chaptermark{Third Chapter Heading}

    Глава направлена на преодоление разрыва между теоретическими моделями и практической реализацией системы управления, обеспечивая устойчивость и надежность в различных подводных сценариях.

\section{Математическое моделирование}

    ТНПА имеют шесть степеней свободы (DOF), в том числе три для перевода и три для вращения. Использование кватернионов для представления ориентации помогает избежать проблем с сингулярностью.

    Движение морского транспортного средства без учета внешних сил, 
    выражается в терминах скоростей для двух координатных систем:
    \begin{equation*}
        \mathbf{\dot{\bar{r}}^N=J(\bar{r}^N) \bar{v}^B}
    \end{equation*}
    Подход Ньютона-Эйлера используется для описания динамики ТНПА, связывая приложенные силы и моменты с линейными и угловыми ускорениями:
    \begin{equation*}
        \mathbf{M \dot{\bar{v}}^B + C(\bar{v}^B) \bar{v}^B+D(\bar{v}^B) 
    \bar{v}^B+g(\bar{r}^N)= \bar{f}^B}
    \end{equation*}
    Дополнительные условия, такие как добавочная масса, центр плавучести и эффекты демпфирования, повышают точность модели.

    Однако эмпирические корректировки модели необходимы из-за сложной, нелинейной и связанной природы динамики ТНПА.
    С учетом динамики, следующие параметры могут иметь некоторую неточность:
    \begin{itemize}
        \item \textbf{Параметры тела}. Матрица масс $M_B$ и матрица Кориолисовых сил $C_B$ могут быть неизвестны из-за неопределенных значений массы $m$ и матрицы инерции $I_0$.
        \item \textbf{Коэффициенты вязкого демпфирования} $D$. Значения линейных и квадратичных членов определяются эмпирически.
        \item \textbf{Восстановительные силы} $g$. В частности, плотность воды $\rho$ зависит от окружающей среды, а объем всего тела $\nabla$ трудно рассчитать с надлежащей точностью.
        \item \textbf{Параметры добавленной массы} $M_A$ и $C_A$. Они не могут быть вычислены напрямую и будут исключены в будущих расчетах.
    \end{itemize}
    
    Моделирование подруливающих устройств определяет желаемую тягу от каждого подруливающего устройства. Сложная взаимосвязь между силой тяги и переменными управления упрощается с помощью матрицы конфигурации тяги $T$ и передаточной функции усиления постоянного тока функции $\phi(u)$:
    \begin{equation*}
        \mathbf{\bar{f}^B=T\boldsymbol{\phi}(u)}
    \end{equation*}
    Предлагается использовать полином третьего порядка $\phi_0(u)$ для моделирования приведенной выше передаточной функции. Этот подход считается подходящим из-за общей формы зависимости ШИМ от тяги. 
    Приближенная передаточная функция выражается как: 
    \begin{equation*}
        \mathbf{\boldsymbol{\phi}(u) = k\boldsymbol{\phi}_0(u)}
    \end{equation*}
    
    Дополнительно опишем внутренние и внешние силы, действующие на 
    тела с помощью единственного нелинейного члена $h(r,v)$:
    \begin{equation*}
        \mathbf{h(r,v) = C(v)v + D(v)v + g(r)}
    \end{equation*}
    Наконец, приближенное уравнение динамики системы определяется как:
    \begin{equation*}
        \mathbf{M \dot{v} + h(r, v) + \boldsymbol{\delta} = kT\boldsymbol{\phi}_0(u)}
    \end{equation*}

\section{Проектирование систем управления}

\subsection{Обратная динамика}

    Метод нелинейного управления, известный как обратная динамика, позволяет отслеживать траекторию движения, рассчитывая крутящие моменты приводов, необходимые для достижения определенной траектории. Этот подход основан на точном устранении нелинейностей в уравнении движения робота.

    \textbf{Дизайн закона управления}: Учитывая приближенное уравнение динамики, мы можем разработать следующий закон управления для линеаризации системы:
    \begin{equation*}
        \mathbf{\boldsymbol{\nu} = \boldsymbol{\phi}_0(u) = \hat{B}^{+}(\hat{M}a + \hat{h}(r, v))}
    \end{equation*}
    
    в то время как внешний контур управления $a$ разработан как пропорционально-дифференцирующий (ПД) регулятор:
    \begin{equation*}
        \mathbf{a = \dot{v}_{des} - K_p \tilde{r} - K_d \tilde{v}}
    \end{equation*}

    Метод обратной динамики, хотя и эффективен в теории, представляет собой ряд проблем в практическом применении:
    \begin{itemize}
    \item \textbf{Неопределенные или большие границы ошибок}: Границы ошибки могут быть большими или неопределенными, если $\dot{v}_{des}$ не ограничена должным образом.
    \item \textbf{Сложность настройки}: Параметры $K_p$ и $K_d$ требуют тщательной настройки для достижения желаемой производительности.
    \item \textbf{Недостаточная робастность}: Метод не устойчив к неопределенности параметров и внешним возмущениям.
    \end{itemize}

    В результате метод обратной динамики может оказаться не лучшим вариантом для эффективного управления подводными системами.

\subsection{Управление в скользящем режиме}

    Управление в скользящем режиме (SMC) обладает рядом преимуществ для управления подводными роботами в неопределенной и динамичной среде.

    Ключевые компоненты SMC включают в себя:
    \begin{itemize}
        \item \textbf{Дизайн поверхности скольжения}: Поверхность скольжения определяется для обеспечения желаемого поведения системы, а закон управления разрабатывается для приведения траектории системы к этой поверхности.
        \begin{equation*}
            \mathbf{s(r, t) = (\frac{d}{dt} + \boldsymbol{\lambda})^{n-1}\tilde{{r}}}
        \end{equation*}
        \item \textbf{Условие скольжения}: Условие скольжения гарантирует, что траектория системы сходится к поверхности скольжения за конечное время.
        \begin{equation*}
            \mathbf{s^T\dot{s} < - \boldsymbol{\eta} \|s\| \text{\quad or \quad} \|s\|\|w\| - s^TKa_s \leq - \boldsymbol{\eta} \|s\|}
        \end{equation*}
        \item \textbf{Дизайн закона управления}: Закон управления сочетает в себе номинальные $a_n$ и робастные $a_s$ компоненты для стабилизации системы и обработки неопределенностей и возмущений.
        \begin{align*} 
            &\mathbf{\boldsymbol{\nu} = \hat{B}^{+}(\hat{M}(a_n + a_s) + \hat h(r, v))} \\
            &\mathbf{a_n = - K_p\tilde{r} - K_d\tilde{v}} \\
            &\mathbf{a_s = 
            \begin{cases}
            \mathbf{\boldsymbol{\rho} \frac{s}{\|s\|}, \quad \|s\| >\boldsymbol{\epsilon}}\\
            \mathbf{\boldsymbol{\rho} \frac{s}{\boldsymbol{\epsilon}}, \quad \|s\| \leq \boldsymbol{\epsilon}}
            \end{cases}}
        \end{align*}
    \end{itemize}

    Используя скользящий режим управления, подводные роботы могут добиться повышенной стабильности, точности и быстроты реакции даже в сложных условиях.

\subsection{Управление на основе оптимизации}

    Управление на основе оптимизации предлагает перспективное решение сложных проблем управления путем поиска оптимальных управляющих воздействий, которые удовлетворяют критериям эффективности, учитывая при этом ограничения системы.

    \textbf{Задача оптимизации} : Квадратичное программирование (QP) - это тип выпуклой оптимизации, в которой целевая функция квадратична, а ограничения линейны.
    Стандартная форма задачи QP такова:
    \begin{equation*}
        \begin{aligned}
            & \mathbf{\underset{x \in R^n}{\text{minimize}}}
            & & \mathbf{\frac{1}{2} x^T Q x + c^T x} \\
            & \text{subject to}
            & & \mathbf{A x \leq b}, \\
            & & & \mathbf{E x = d}
        \end{aligned}
    \end{equation*}

    \textbf{Дизайн закона управления}: Оптимизация скользящего управления может быть сформулирована как QP-задача:
    \begin{equation*}
    \begin{aligned}
    \mathbf{\min_{a_s, \boldsymbol{\nu}}} \quad & \mathbf{a_s^T R_a a_s  + \boldsymbol{\nu}^T R_{\boldsymbol{\nu}} \boldsymbol{\nu} +
    \gamma_0 d^2 + \gamma_1 \|a_s - a_{s (prev)}\|} \\
    \textrm{s.t.} \quad & \mathbf{s^TKa_s \geq \boldsymbol{\eta} \|s\| + \|s\|\|w\| + d} \\\
    & \mathbf{\hat Ma_s - \hat B\boldsymbol{\nu} = -(\hat{M}a_n + h(r, v))} \\
    & \mathbf{\boldsymbol{\phi}_0(u_{min}) \leq \boldsymbol{\nu} \leq \boldsymbol{\phi}_0(u_{max})}
    \end{aligned}
    \end{equation*}
    где для уменьшения дребезга вводится элемент неопределенности $d$,
    и добавляется слагаемое $\|a_s - a_{s (prev)}\| $ для сглаживания сигнала управления на выходе.

    Таким образом, управление на основе оптимизации обеспечивает надежную основу для решения сложных проблем управления, используя методы оптимизации для получения законов управления, включающих динамику системы, ограничения и цели для достижения оптимальной производительности и адаптивности.