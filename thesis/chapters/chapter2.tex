\chapter{Literature Review}

Nowadays robots and humans work spatially close together. In this environment safety is a major concern, since a robot could potentially be the cause of multiple injury threats, including electrical, mechanical or even injuries caused by a collision with the robot \cite{eval_of_safety}. This concern entailed works that evaluate theoretical injuries, their classifications and possible outcomes \cite{safety}. The safety of a human when working in close proximity of a robot can be achieved by both robot's actuator design and by methods of force factors estimation, such as collision detection \cite{collision}, collision reaction \cite{coll_reac} or real-time collision avoidance \cite{coll_avoid}. A considerable amount of injuries are a result of poor actuator design \cite{intrinsic} as the safety of the machine is reliant on the actuator's stiffness, gear ratio and impedance and therefore robots like Puma 560 require significant cushioning to render it safe, as discussed in \cite{unsafe_manip}. 

Therefore, to guarantee overall safety of the system with the smallest transmission ratio possible and with low impedance - an appropriate actuator design should be chosen for the robotic system.


The design of the actuator should imply safety for a human. The possible large impact loads resulting from a close proximity to a robotic system with high impedance poses a risk of serious mechanical injuries \cite{unsafe_manip}. Hence, safety requires low impedance of the actuator.
One way to achieve this is to design the system with the Series Elastic Actuator (SEA). SEA is a system, where a load is connected to the actuator in series with an elastic element, for example with springs. SEAs possess low impedance at all frequencies \cite{sea}. While most electric motors use gear reduction to support heavy loads, which in turn creates high impedance, SEAs help to alleviate the disadvantages of gear usage, namely reducing peak gear forces, thus allowing for a stable force control, as well as increasing shock tolerance.
Moreover, some improvements for SEAs have been proposed, such as creating an inner position loop and using remote sensors for increased proportional gain, better force fidelity, and higher controllable bandwidths \cite{sea_improvs}.


Another solution for the problem of the need for powerful actuators in even the simplest designs is to introduce Twisted String Actuators (TSA) to the system.
TSA is a system consisting of cables that contract as a result of being twisted by a motor, in turn creating linear displacement due to the contraction of the string. TSAs are cheap, very light and most importantly, they offer a high (although nonlinear) reduction ratio that decreases with the increasing twist angle \cite{TSA_math_model}. Moreover, unlike SEAs, with a proper choice of both the string and the motor TSAs can achieve near constant transmission ratio that prove TSAs to be a considerably better choice. Moreover, TSAs provide the possibility of controlling the system using only force feedback as reported in \cite{TSA_intro}. In fact, no measurement of the system state (motor angular displacement or string's linear displacement and their respective velocities) neither accurate values of the system parameters (e.g. string length, string radius) are required.
These advantages are the main reason that a TSA is often integrated in the design of many robotic systems, such as the gripper design \cite{gripper}, in designs of exoskeletons and exosuits \cite{exo}, as well as in similar systems that often interact with a human.

The cases mentioned above are a few examples of such systems that cooperate with a human or are piloted by one. Either way that person would exert some force onto the robot, causing its dynamics to alter. In general, HMI is a common cause of external torques occurring at the end-effector of a robotic system. Therefore, it is necessary for a robotic configuration to have a system for evaluating the external torque. Moreover, it is desirable for a robotic system to be sensorless because of lower cost, lightness and subsequently increased mobility of the end-effector. As mentioned before, TSAs are simple in their design and therefore they enable a simple control design. These important aspects allow for implementation of a force estimation algorithm by using only proprioceptive measures (motor angular displacement or string's linear displacement and their respective velocities) in a TSA robotic system. By choosing both a safe and efficient actuator as well as designing the estimation and control of the force factors in this cable-driven robotic system we can insure safety in HMI.


After choosing an actuator, an appropriate observer should be applied to the system. Therefore, in this work we implement several control algorithms on an experimental setup and compare their estimations. The considered algorithms  applied to a cable-drive system are the following:
\begin{enumerate}
    \item sensorless estimation based on the increase of the motor current;
    \item Classic Momentum Observer \cite{momentum_obs};
    \item Sliding Mode Disturbance Observer \cite{sliding_mode}.
\end{enumerate}
% NEED CITATION AND MORE INFO
The first and the most straightforward method of collision detection is by observing the changes of motor current. It is observed that the current at the motor increases, when the end-effector comes in contact with an object or during HMI, resulting in an acquired external torque and increased load on the system. Similarly to other devices, assuming constant voltage at the motor, the larger load causes the motor current to increase. This correlation between them is used to design a control model performed without any force sensors.

% MORE KEYPOINTS
However, this method is rather inaccurate and provides only a rough estimation of external torques. Therefore, to have a better force estimation and control a control model was proposed - the classic Momentum Observer \cite{momentum_obs}. They consider a HMI as a faulty behaviour and solve it using only proprioceptive sensors. Their method is based on the generalized momentum and its residual vector, which acts as a filtered version of the
joint torques resulting from cartesian contact forces. The components of this residual vector contain information on the collision and its location. 'Moreover the residual vector can be used to decompose the joint velocity space into complementary directions, so that a hybrid force/motion controller can be designed'. The observation of the system's generalized momentum and its residual was given many applications, such as collision detection \cite{collision} and friction observers \cite{fric_obs}. %should read on that.

% ALSO MOAR keyPOINTs
However, it is observed in the study by Gianluca Garofalo et al. \cite{sliding_mode} that the classic disturbance observer provides an exact estimation model only in case where the external torques are constant. Additionally, this model's precision suffers from inaccuracies in physical measurements, so two new observers capable of estimating any external torques are proposed. The Sliding Mode Disturbance Observers are a modified and improved force control model and are capable of estimating any external torques without requiring additional hardware for the robot. These new observers insure convergence of the estimated torques to the real ones in finite time. Additionally they are capable of a finite-time estimation of the joint acceleration. This finite-time behavior of new observers is derived using sliding mode control \cite{slid_mode_control}



With this in mind, this work aims to implement collision detection in HMI or during collision with an unknown object using several disturbance observers using an experimental setup based on the TSA. The results of the experiments will help to determine which force estimation method fits best to a TSA system, which to the best of our knowledge has not been done before. Our implementation considerably simplifies the design and cheapens the production of cable-driven robotic systems by substituting one of their most expensive parts - force sensors - with a force estimation algorithm, hence further justifying the importance and novelty of this work. \\