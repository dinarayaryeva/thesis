\chapter{Literature Review}
\label{chap:lr}

\chaptermark{Second Chapter Heading}

    This chapter presents a comprehensive overview of existing research on remotely operated vehicles (ROVs), focusing on modeling and control design. This literature review aims to investigate and synthesize control strategies, addressing uncertainties from actuator dynamics and environmental factors, which correspond to the proposed research question.

    Section 1 gives a brief description of ROVs, including their types and general attributes. Section 2 explains the commonly accepted assumptions and mathematical models used in the field. Section 3 offers an extensive review of various studies on the application of robust control design for underwater systems. In conclusion, Section 4 summarizes the key insights from the review and suggests a control methodology.

\section{Overview on ROVs}

    This section provides an overview of ROVs, from their classification and applications to their inherent characteristics and challenges. The obtained knowledge forms a basis for ROV modeling and control design.

    According to \cite{rov_review}, an ROV's mechanical structure consists of a monitoring camera, a sensor for gathering navigation data, and actuators for directional control. A comparative study \cite{overview2} found that the physical aspects that affect ROV functions are the accuracy of sensor systems and the thruster designs. The unpredictable nature of underwater currents, drag,
    and buoyancy dynamics can also have a serious impact on a ROV's performance, complicating model design.

    Several studies \cite{rov_application, overview} have identified that ROVs have become crucial for industrial applications,
    offshore oil and gas exploration, patrolling, and surveillance. Therefore, its control system
    should focus on position tracking and station keeping in the presence of parameter and environmental uncertainties, addressing the following issues.

    The subsequent section explores the mathematical models that serve as the basis for designing 
    such control systems. These models are crucial for comprehending and predicting 
    the behavior of ROVs under varying operating conditions.

\section{Mathematical Modelling}

    Remotely operated vehicles require mathematical models for various purposes, including control
    system design, simulation, and performance analysis.

    Thor I. Fossen \cite{fossen:guidance} provided a complete description of the fundamentals of mathematical modeling for marine
    vehicles. In his book, ROV was represented as a single rigid body (SRB) by considering it as a solid mass with no internal movement or deformation. The SRB model has drastically simplified the modeling while capturing the essential dynamics of the system. 

    From a kinematic point of view, ROVs have six degrees of freedom (DoFs). However, the orientation expressed in the rotation angles could eventually lead to the singularity. To solve this issue,
    \cite{quat_smc} proposed the quaternion representation. The dynamics was derived based on classical physics laws: Newton's Second Law and Euler-Lagrange equation, forming the set of nonlinear equations. The study \cite{identification} simplified the equations and represented them in a matrix form.

    However, several sources have established that some aspects of the ROV dynamics require empirical estimates due to their complex, nonlinear, and coupled nature \cite{fossen:guidance, bluerov}. For instance, the inertia of the surrounding fluid cannot be neglected when the vessel moves through a viscous medium. Additionally, water damping is another source of nonlinearity that can be approximated as a function of the velocity. Finally, aside from buoyancy and
    gravity, it is common practice to cancel all other forces acting on a vehicle, although it can also impact the dynamics \cite{bluerov}.

    Moreover, thruster modeling must be applied to define the desired thrust of each thruster. A recent study by \cite{bluerov} stated that creating an accurate thruster model can be challenging due to the influence of factors such as motor models, hydrodynamic effects, and propeller mapping.
    
    By making these simplifications, the control system cannot independently provide effective control over such uncertain dynamics. As a result, a robust controller design is necessary for precise ROV position tracking.

\section{Control Solutions}

    Controlling a remotely operated vehicle (ROV) is a complex task that involves several processes to stabilize the vehicle and ensure it follows the operator's instructions. To guarantee the 
    system's robustness, it's essential to establish a control system capable of managing disturbances caused by changes in parameters and the environment.

    According to \cite{overview, control_overview}, ROV control faces two main challenges: unmodeled elements like added mass and hydrodynamic coefficients, and highly nonlinear dynamics of the underwater environment causing significant disturbances for the vehicle.

    One of the most basic approaches for nonlinear control is the inverse dynamics approach \cite{spong_book}. Despite its simplicity, this approach is limited by its reliance on precise model parameters and its inability to handle significant disturbances effectively.

    The research on ROV control showed several schemes that can achieve robust stability under variable disturbances. The classical approach applied is sliding mode control (SMC), which was
    introduced by Utkin \cite{utkin}. SMC is an effective way to address the issues mentioned above and is, therefore, a feasible option for controlling underwater vehicles. However, standard SMC introduces high-frequency signals, which can cause actuator switching and consequently decrease its lifetime \cite{slotine, spong}.

    Through the application of convex optimization, a balance can be achieved between control performance and actuator longevity, showing promise in enhancing the reliability and efficiency of control systems \cite{utkin_opt, utkin_book}. This method allows for systematic adjustments of SMC parameters to attain optimal 
    performance, thereby strengthening robustness against disturbances and addressing high-frequency switching issues.

\section{Summary}

    This literature review comprehensively examined the general characteristics, mathematical modeling,
    and control solutions for underwater vehicles. The overview identified a gap in the field of robust
    control for remotely operated vehicles, particularly in dealing with parameter and environmental
    uncertainties. While the original sliding mode control scheme provided robust stability, it suffered
    from high-frequency output oscillations. To address this issue, a combination of convex optimization and
    sliding mode was proposed.