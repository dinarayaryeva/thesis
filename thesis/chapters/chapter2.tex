\chapter{Literature Review}
\label{chap:lr}

This chapter presents a comprehensive overview of existing research on remotely operated vehicles (ROVs), 
focusing on modeling and control design. This literature review aims to investigate and synthesize 
control strategies, addressing uncertainties from actuator dynamics and environmental factors, which 
correspond to the proposed research question.

Section 1 gives a brief description of ROVs, including their types and general attributes. 
Section 2 explains the commonly accepted assumptions and mathematical models used in the field. 
Section 3 offers an extensive review of various studies on the application of robust control 
design for underwater systems. In conclusion, Section 4 summarizes the key insights from the review 
and suggests a control methodology.

\section{Overview on ROVs}

This section provides an overview of ROVs, from their classification and applications to their 
inherent characteristics and challenges. The obtained knowledge forms a basis for ROV modeling 
and control design.

According to \cite{rov_review}, an ROV's mechanical structure consists of a monitoring camera, a sensor 
for gathering navigation data, and actuators for directional control. A 
comparative study \cite{overview} found that the physical aspects that affect ROV functions are the accuracy of 
sensor systems and the thruster designs. The unpredictable nature of underwater currents, drag, 
and buoyancy dynamics can also have a serious impact on a ROV's performance, complicating model 
design.

Several studies \cite{rov_review, overview} have identified that ROVs have become crucial for industrial applications, 
offshore oil and gas exploration, patrolling, and surveillance. Therefore, its control system 
should focus on position tracking and station keeping in the presence of parameter and environmental uncertainties, addressing the following issues.

A recent systematic review \cite{inspection_review} concluded that there are two primary classifications for ROVs based 
on their functions and intended use: observation class and work class. Observation class vehicles 
are typically small and limited to shallow waters with propulsion power up to a few kilowatts. Work 
class vehicles can perform heavy-duty work, requiring significant hardware 
system complexity. Thus, when the functionality of these large 
ROVs is not necessary, a smaller ROV is preferred for a wide range of applications.

\section{Mathematical Modelling}

Remotely operated vehicles require mathematical models for various purposes, including control 
system design, simulation, and performance analysis.

Fossen \cite{fossen:guidance} provided a complete description of the fundamentals of mathematical modeling for water 
vehicles. In this book, ROV was represented as a 
single rigid body. The SRB model has drastically simplified the model while capturing the essential 
dynamics of the system. From a kinematic 
point of view, ROVs have six degrees of freedom (DoFs). However, the orientation expressed 
in the rotation angles could eventually lead to the singularity. To solve this issue, 
\cite{quat_smc} proposed the quaternions representation. The dynamics were derived based on classical 
physics laws: Newton’s Second Law and Euler-Lagrange equation, forming the set of nonlinear equations.
The study \cite{identification} simplified the similar dynamics and represented it in a matrix form.

However, several sources have established that some aspects of the ROV dynamics require empirical 
estimates due to their complex, nonlinear, and coupled nature \cite{fossen:guidance, bluerov}. For instance, the inertia 
of the surrounding fluid cannot be neglected when the vessel moves through a viscous medium. Additionally, water damping is another source of nonlinearity that 
can be approximated as a function of the velocity. Furthermore, aside from buoyancy and 
gravity, it is common practice to cancel all other forces acting on a vehicle, although it can also impact the dynamics \cite{bluerov}.

Moreover, thruster modeling must be applied to define the desired thrust of each thruster. A recent 
study by \cite{bluerov} stated that creating an accurate thruster model can be challenging due to the 
influence of factors such as motor models, hydrodynamic effects, and propeller mapping.

By making these simplifications, the control system cannot independently provide effective control 
over such uncertain dynamics. Therefore, a robust controller design is necessary for precise ROV 
position tracking.

\section{Control Solutions}

Controlling an ROV is a complex task that requires a set of processes to stabilize the vehicle and 
to make it follow the operator's instructions. To ensure the robustness of the system, it is 
necessary to define a control system that can handle disturbances caused by parameter and 
environmental changes.

According to \cite{overview}, there are two main challenges associated with ROVs control:
\begin{enumerate}
\item Unmodeled elements like added mass and hydrodynamic coefficients.
\item Highly nonlinear dynamics of the underwater environment which cause significant disturbances to the vehicle.
\end{enumerate}

The recent research on ROV control showed several schemes that can achieve robust stability under 
variable disturbances. The classical approach applied is sliding mode control (SMC), which was 
introduced by Slotine \cite{slotine}. SMC is an effective way to address the issues mentioned above and is, 
therefore, a feasible option for controlling underwater vehicles. However,
standard SMC introduces high-frequency signals, which can cause actuator switching and consequently 
decrease its lifetime.

The modern SMC interpretations keep the main advantages, 
thus removing the chattering effects. One of the possible approaches is Adaptive Sliding Mode Control 
(ASMC) design. \cite{fossen:control} designed an effective hybrid control mechanism for the underwater system that 
follows a given route, adapting to dynamic disturbances. \cite{adaptive_smc} improved a control system for ROVs, 
eliminating the need for dynamics linearization. Another way to refine SMC is to add an integral 
component into the controller equation. The proposed Integral Sliding Mode Control (ISMC) reduced 
chattering, effectively eliminating the uncertainty of the model parameters \cite{integral_smc}.

The choice of control strategy should be based on the specific 
requirements and characteristics of the underwater vehicle. While both SMC modifications showed 
precise control, these approaches faced certain challenges \cite{integral_smc}, \cite{adaptive_smc}. The ISMC method had limitations in adapting to rapidly changing dynamics, while the ASMC method experienced potential trade-offs in transient performance. 
These problems were opposite to each other. Therefore, combining them into Adaptive Integral Sliding 
Mode Control (ADISMC) can be a good idea to leverage the strengths of both methods and compensate 
for their weaknesses.

\section{Summary}

This literature review comprehensively examined the general characteristics, mathematical modeling, 
and control solutions for underwater vehicles. This overview identified a gap in the field of robust 
control for remotely operated vehicles, particularly in dealing with parameter and environmental 
uncertainties. While the original sliding mode control scheme provided robust stability, it suffered 
from high-frequency output oscillations. There were several variations of SMC available, and each 
algorithm has its limitations. To address this research gap, a combination of adaptive and integral 
sliding mode modifications was proposed.