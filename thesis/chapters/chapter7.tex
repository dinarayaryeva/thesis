\chapter{Conclusion}
\label{chap:conclusion}

This thesis addresses the challenge of controlling Remotely Operated Vehicles (ROVs) in underwater environments with nonlinearities and uncertainties. It focuses on enhancing Sliding Mode Control (SMC) using convex optimization techniques to improve stability, performance, and reduce the chattering effect.

The research's key contributions include developing a comprehensive mathematical model of an ROV and creating an advanced SMC scheme integrating convex optimization. The proposed control scheme was validated through simulations, showing improved trajectory tracking and disturbance rejection compared to traditional methods.

The findings highlight the improved stability and performance of the enhanced SMC scheme, providing better reliability and efficiency in the ROV's operation. The reduction of chattering was achieved using convex optimization, enhancing control performance and reducing wear and tear on the ROV's components.

Areas for further investigation include enhancing the computational efficiency of the convex optimization process and testing the proposed control strategy in diverse underwater environments. Additionally, future research could explore integrating advanced sensing technologies to enhance the adaptability and intelligence of ROV control systems.

In summary, this thesis significantly contributes to the field of underwater robotics by providing a robust and efficient control strategy for ROVs. The integration of convex optimization with sliding mode control offers a solution that balances performance, stability, and robustness, supporting the development of more capable and reliable underwater robotic systems in challenging environments.


