\chapter{Mathematical Modelling}
\label{chap:mat}

\chaptermark{Third Chapter Heading}

    Remotely operated vehicles (ROVs) are complex systems that require mathematical models for various purposes, 
    including control system design, simulation, and performance analysis. With accurate mathematical models, 
    ROVs are able to navigate through different underwater terrains and complete control tasks with a good 
    precision. Also, the simulation, based on these models, are suitable to test different work scenarios and 
    detect undesirable ROV's behavior before the physical experiment.   
    
    The fundamentals of the modelling for marine vehicles were fully described in Fossen \cite{fossen:guidance}.
    Using common assumptions, ROV is treated as a single rigid body with six degrees of freedom (DOF).
    By considering the vehicle as a rigid body, we can simplify the mathematical modeling process while capturing the essential dynamics 
    of the system.

    In order to effectively model rigid bodies, it is crucial to consider their kinematic and dynamic properties.

    \section{Notations}
    
    Before examining the theory underlying the modeling process, 
    it is crucial to establish clarity on the notations. 

    In the context of motion with six degrees of freedom (DOF), 
    a set of six independent coordinates is defined within the 
    coordinate frame: three for translational directions 
    (surge, sway, and heave) and three for rotational directions 
    (roll, pitch, and yaw), as shown in \pic{image:6dof}.

    \begin{figure}[H]
        \centering\includegraphics*[width=0.4\textwidth]{6dof}
        \caption{Illustration of the Six Degrees of Freedom (DOF) for a Marine Vehicle}
        \label{image:6dof}
    \end{figure}

    The linear position of the body in three-dimensional space is defined as 
    $ \mathbf{r} = [r_x, r_y, r_z]^\top $, representing translations along 
    the x, y, and z axes, respectively. 
    
    While the orientation of the body can 
    be expressed using Euler angles, this approach can lead to singularities 
    when the sway angle approaches $\pm 90^\circ $. To overcome this limitation, 
    quaternions provide a more robust representation by introducing redundancy. 
    A quaternion, represented in scalar-first form, is defined as 
    \begin{equation}
        \mathbf{q} = q_0 + q_1i + q_2j + q_3k = [q_0, q_1, q_2, q_3]^\top
        \label{eq::quat}
    \end{equation} 
    This quaternion-based representation avoids the singularities associated with 
    Euler angles and ensures a smooth and continuous description of orientation 
    in three-dimensional space.

    The velocity vectors can be defined separately: $\mathbf{v} = [v_x, v_y, v_z]^\top $ for translation along the x, y, and z axes, 
    and $ \boldsymbol{\omega} = [\omega_x, \omega_y, \omega_z]^\top$ for rotation around 
    the same axes. 
    Similarly, this applies to linear forces and torques, where $\mathbf{f}$ represents 
    the linear forces and $\boldsymbol{\tau}$ represents the torques acting on the body.

    \begin{table}[H]
    \captionsetup{justification=centering}
    \caption{Notations for Motion Parameters in ROV Modeling}
    \label{table:notation}
    \begin{center}
    \begin{tabular}{ccc}
        \hline Symbol & Description & Dimensionality\\
        \hline
        $r$ & Linear position vector & $\mathbb{R}^{3}$ \\
        $q$ & Angular position (orientation) vector & $\mathbb{R}^{4}$\\
        $v$ & Linear velocity vector & $\mathbb{R}^{3}$\\
        $\omega$ & Angular velocity vector & $\mathbb{R}^{3}$\\
        $f$ & Vector of linear forces& $\mathbb{R}^{3}$ \\
        $\tau$ & Vector of torques& $\mathbb{R}^{3}$ \\
        \hline
        \end{tabular}
    \end{center}
    \end{table}

    For convenience, it is desirable to define combined vectors for positions, 
    velocities, and forces as follows:
    \[ \bar{r} = \begin{bmatrix}
        r \\
        q
    \end{bmatrix}, \quad \bar{v} = \begin{bmatrix}
        v \\
        \omega
    \end{bmatrix}, \quad \bar{f} = \begin{bmatrix}
        f \\
        \tau
    \end{bmatrix} \]

    To manipulate the obtained vectors, it is necessary to define cross product 
    operators. For vectors in \( \mathbb{R}^3 \), the cross product is represented 
    by multiplication with a skew-symmetric matrix \( S(\lambda) \):
    \begin{equation}
        S(\lambda) = \begin{bmatrix}
            0 & -\lambda_3 & \lambda_2 \\
            \lambda_3 & 0 & -\lambda_1 \\
            -\lambda_2 & \lambda_1 & 0
        \end{bmatrix}
    \end{equation} 
    For vectors in \( \mathbb{R}^6 \), the cross product operator 
    \( \bar{\times}^* \) is defined as:
    \begin{equation}
        x \bar{\times}^* = \begin{bmatrix}
            S(x_2) & 0_{3 \times 3} \\
            S(x_1) & S(x_2)
        \end{bmatrix}
        \text{ for } x = \begin{bmatrix}
            x_1 \\
            x_2 
        \end{bmatrix}
    \end{equation}

\section{Frames of reference}

    In order to determine the kinematics and dynamics of the system, it is essential 
    to project the calculations into a consistent frame of reference. 
    In some cases, multiple coordinate frames are established based on the system's configuration.
    
    For ROV, it is reasonable to define two coordinate frames. 
    These frames are the earth-fixed frame, which is inertial with fixed origin, and the body-fixed frame, 
    which is a moving frame attached to the vehicle as depicted in \pic{image:frames}. 
    \begin{figure}[H]
        \centering\includegraphics*[width=0.5\textwidth]{frames}
        \caption{Coordinate Frames Used in the Kinematic and Dynamic Modeling of ROVs}
        \label{image:frames}
    \end{figure}

    The origin of the body-fixed frame usually coincides with the vehicle's center of mass,
     and its axes are chosen along the vehicle's principle axes of inertia.

    The state variables of the rigid body expressed in the body-fixed frame would be 
    denoted by $^B$ and in the earth-fixed frame by $^N$.

\section{Kinematics}

    Kinematics describes the motion of the marine vehicle without considering the forces acting upon it.
    In order to describe kinematic motion of the body, it is necessary to find relation between velocities in two coordinate frames.
    This relation can be represented with linear transformations as:
    \begin{equation}
    \begin{aligned}
        & \dot{\bar{r}}^N=J(\bar{r}^N) \bar{v}^B \\
        \text{where } & J(\bar{r}^N)=\left[\begin{array}{cc}
        R(\bar{r}^N) & 0_{3 \times 3} \\
        0_{4 \times 3} & T(\bar{r}^N)
        \end{array}\right]
    \end{aligned}
    \label{eq::kinematics}
    \end{equation}
    
    The rotational matrix $R$ and the transformation matrix $T$ using quaternion representation \ref{eq::quat} can be expressed as follows:
    \begin{equation}
    R(q)=\left[\begin{array}{ccc}
        1-2\left(q_2^2+q_3^2\right) & 2\left(q_1 q_2-q_3 q_0\right) & 2\left(q_1 q_3+q_2 q_0\right) \\
        2\left(q_1 q_2+q_3 q_0\right) & 1-2\left(q_1^2+q_3^2\right) & 2\left(q_2 q_3-q_1 q_0\right) \\
        2\left(q_1 q_3-q_2 q_0\right) & 2\left(q_2 q_3+q_1 q_0\right) & 1-2\left(q_1^2+q_2^2\right)
        \end{array}\right]
    \end{equation}
    \begin{equation}
    T(q)=\frac{1}{2}\left[\begin{array}{rrr}
        -q_1 & -q_2 & -q_3 \\
        q_0 & -q_3 & q_2 \\
        q_3 & q_0 & -q_1 \\
        -q_2 & q_1 & q_0
        \end{array}\right]
    \end{equation}

\section{Dynamics}

    The Newton-Euler approach is commonly used to describe the dynamics of marine vehicles.
    This approach relates the applied forces and moments to the vehicle's linear and angular accelerations.
    The general equation of motion using the Newton-Euler approach in the body-fixed frame can be written as:
    \begin{equation}
        M\dot{\bar{v}}^B+\bar{v}^B\bar{\times}^*M\bar{v}^B=\bar{f}^B
    \end{equation}
    where $M$ represents the inertia matrix of the rigid body.

    The equation above can be further transformed into standard manipulator equation form:
    \begin{equation}
     M_B \dot{\bar{v}}^B+C_B(\bar{v}^B) \bar{v}^B
    = \bar{f}^B
    \end{equation}
    \begin{conditions}
        M_B & the rigid body mass matrix $\in \mathbb{R}^{6 x 6}$\\
        C_B & the rigid body Coriolis and centripetal forces' matrix $\in \mathbb{R}^{6 x 6}$ 
    \end{conditions}
    \vspace{0.3cm}
    Nevertheless, some additional terms should be included in the equation to determine the specifics of the ROVs model. 
    These terms comprise added mass, which represents the inertia of the surrounding fluid, the shift of the 
    center of buoyancy due to changes in trim and heel angles, and damping effects. 
    By incorporating these terms into the manipulator equation derived from the Newton-Euler approach, 
    the model becomes more accurate and reflects the natural behavior of the ROV.    

\subsection{Center of Gravity and Center of Buoyancy}

    Due to the robust design of the marine vehicles, the center of buoyancy (COB) is usually aligned with
    the center of mass (COM), but placed higher.
    This shift between centers causes torque acting against the capsizing (Fig. \ref{image:scheme}).
    \begin{figure}[H]
        \centering\includegraphics*[width=0.5\textwidth]{01_srb}
        \caption{Schematic representation of the marine vehicle showing the center of buoyancy (COB) and the center of mass (COM).}
        \label{image:scheme}
    \end{figure}

    If we place the origin of the body frame at the center of mass, the mass matrix can be expressed as:
    \begin{equation}
        M_B=\left[\begin{array}{cc}
            m I_{3 \times 3} & -m S\left(r_G^B\right) \\
            m S\left(r_G^B\right) & I_0
        \end{array}\right]
    \end{equation}
    where $r_G^B$ is the vector of the gravity center in the body frame, that is eventually a zero vector.

\subsection{Concept of added mass}

    Since the vehicle moves in a viscous environment, we can not neglect the inertia of the surrounding liquid.
    To compensate added mass effect, it is necessary to add two components into dynamics equation: 
    the added mass and the Coriolis forces acting on the added mass.

    We can define vector of dynamical parameters of our body as:
    \begin{equation}
        f_{\dot{v}} \triangleq \frac{\partial \bar{f}}{\partial \dot{\bar{v}}}
    \end{equation}

    Consequently, the added mass matrix $M_A$ and 
    the Coriolis forces matrix for added mass $C_A(\bar v^B)$
    can be expressed as: 
    \begin{equation}
        M_A=\left[\begin{array}{cc}
            A_{11} & A_{12} \\
            A_{21} & A_{22}
            \end{array}\right]=-\operatorname{diag}\left\{f_{\dot{v}}\right\}, \textrm{where } A_{ij} \in \mathbb{R}^{3 x 3}
    \end{equation}
    \begin{equation}
        C_A(\bar{v}^B)=\left[\begin{array}{cc}
            0_{3 \times 3} & -S\left(A_{11} v^B+A_{12} \omega^B\right) \\
            -S\left(A_{11} v^B+A_{12} \omega^B\right) & -S\left(A_{21} v^B+A_{22} \omega^B)\right.
            \end{array}\right]
    \end{equation}

    The values of dynamical parameters are usually determined
    empirically. Therefore, the error on $M_A$ and $C_A$ can be quite large, and we will not consider
    these matrices in our model implementation for the control design.

\subsection{Hydrodynamic Damping}

    Generally, the dynamics of underwater vehicles can be highly nonlinear and coupled.
    Nevertheless, during the slow non-coupled motion the damping can be approximated to linear and quadratic damping:
    \begin{equation}
        D(\bar{v}^B)=-K_{0} - K_{1}\lvert \bar{v}^B \rvert
    \end{equation}

    The appropriate values of damping coefficients for vectors $K_{0}$ and $K_{1}$ can be discovered through several experiments.

\subsection{Restoring forces}

    The common sense is to neglect all other forces acting on the vehicle except buoyancy and gravity. 
    Although the motion of the current can also affect the dynamics, it is unpredictable and highly nonlinear, 
    which makes it easier to compensate through control.

    The weight of the body is defined as: $\mathbf{W=mg}$, where $m$ is the vehicle's mass and $g$ is the gravity acceleration. 
    The buoyancy force is defined as: $\mathbf{B=\boldsymbol{\rho} g V}$, where $\rho$ 
    is the water density and $V$ the volume of fluid displaced by the vehicle. 
    
    By transforming the weight and buoyancy force to the body-fixed frame, we get:
    \begin{equation}
        f_G\left(\bar{r}^N\right)=R^{\top}\left(\bar{r}^N\right)\left[
        0, 
        0, 
        W
        \right]^\top
    \end{equation}
    \begin{equation}
        f_B\left(\bar{r}^N\right)=-R^{\top}\left(\bar{r}^N\right)\left[
        0, 
        0, 
        B
        \right]^\top
    \end{equation}
    Therefore, overall restoring force and moment vector is defined as:
    \begin{equation}
        g(\bar{r}^N)=-\left[\begin{array}{c}
        f_G(\bar{r}^N)+f_B(\bar{r}^N) \\
        r_G^B \times f_G(\bar{r}^N)+r_B^B \times f_B(\bar{r}^N)
        \end{array}\right]
    \end{equation}
    where $r_B^B$ is the vector of the buoyancy center in the body frame. 

\subsection{Matrix representation}

    Combining all the equations above, the final representation of the system dynamics is:
    \begin{equation}
        M \dot{\bar{v}}^B + C(\bar{v}^B) \bar{v}^B+D(\bar{v}^B) \bar{v}^B+g(\bar{r}^N)= \bar{f}^B
        \label{eq::matrix_dynamics}
    \end{equation}
    \begin{conditions}
        \text{\quad} M & $M_B+M_A=$ the total mass matrix \\
        C(\bar{v}^B) &$C_B(\bar{v}^B)+C_A(\bar{v}^B)=$ the Coriolis and centripetal forces matrix \\
        D(\bar{v}^B) & the hydrodynamic damping \\
        g(\bar{r}^N) & the restoring forces and moments
    \end{conditions}

\section{Thrusters modelling}

    The force and moment vector produced by the thruster are typically represented 
    by a complex nonlinear function $f(\bar{v}^B, V, u)$, which depends on the 
    vehicle's velocity vector $\bar{v}^B$, the power source voltage $V$, and the 
    control variable $u$. 
    % #TODO add sources

    However, expressing such a nonlinear relationship directly can be challenging 
    in practical applications. As a result, some authors propose a simpler approach:
    \begin{equation}
        \bar{f}^B=T\phi(u)
        \label{eq::thrust_model}
    \end{equation}
    \begin{conditions}
        \text{\quad} T & the thrust configuration matrix, which maps body torques to truster 
        forces $\in \mathbb{R}^{6 x n}$ \\
        \phi(u) & the DC-gain transfer function, which defines relation between
        PWM signal and output force $\in \mathbb{R}^{n x n}$
    \end{conditions}

\section{Summary}

This chapter provides a comprehensive overview of essential 
concepts and equations crucial for understanding the kinematics and dynamics of ROVs.

\textbf{Notation} \\
ROVs have six degrees of freedom (DOF), which include three for translation and three for rotation.
The use of quaternions to represent orientation helps to avoid singularity issues.

\textbf{Frame of reference} \\
There are two coordinate frames: the earth-fixed frame and the body-fixed frame. 
State variables are denoted by $_B$ in the body-fixed frame and $_N$ in the earth-fixed one.

\textbf{Kinematics} \\
The motion of a marine vehicle, without considering external forces, 
is expressed in terms of velocities for two coordinate frames (\ref{eq::kinematics}):
$$
\mathbf{\dot{\bar{r}}^N=J(\bar{r}^N) \bar{v}^B}
$$

\textbf{Dynamics} \\
The Newton-Euler approach is used to describe the dynamics of ROV by relating applied 
forces and moments to linear and angular accelerations (\ref{eq::matrix_dynamics}):
$$
\mathbf{M \dot{\bar{v}}^B + C(\bar{v}^B) \bar{v}^B+D(\bar{v}^B) 
\bar{v}^B+g(\bar{r}^N)= \bar{f}^B}
$$
Additional terms like added mass, center of buoyancy, and damping effects enhance model accuracy.

\textbf{Thruster modelling} \\
The complex relationship between thruster force and control variables 
is simplified using a thrust configuration matrix $T$ and a DC-gain transfer 
function $\phi(u)$ (\ref{eq::thrust_model}):
$$
\mathbf{\bar{f}^B=T\boldsymbol{\phi}(u)}
$$

The mathematical models developed in this chapter lay the foundation for designing control systems. 
By accurately capturing the dynamics and kinematics of ROVs, these models enable precise navigation 
underwater. The next chapter will implement effective control strategies and enhance the 
performance of ROVs in various scenarios.