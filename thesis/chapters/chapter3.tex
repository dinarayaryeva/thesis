\chapter{Mathematical Model}
\label{chap:lr}
\chaptermark{Second Chapter Heading}

\section*{Modelling}

    Remotely operated vehicles (ROVs) are complex systems that require accurate mathematical models for various purposes, including control system design, simulation, and performance analysis.
    In order to effectively model these vehicles, it is crucial to consider their rigid body dynamics, frames of reference, kinematics, and dynamics.\\
    ROV is typically treated as a single rigid body in six degrees of freedom (DOF). 
    The six DOF represent the vehicle's motion in three translational directions (surge, sway, and heave) and three rotational directions (roll, pitch, and yaw). 
    By considering the vehicle as a rigid body, we can simplify the mathematical modeling process while capturing the essential dynamics of the system.
    The fundamentals of the modelling for marine vehicles as a rigid bodies were described in Fossen().
    
    \subsection*{Notations}

    Before the start of all derivations, it is reasonable to ensure the use of common notations.
    For the motion with six DOF, it is necessary to define six independent coordinates.
    Futher explanations in this work will be based on the notation of The Society of Naval Architects and Marine Engineers (SNAME) introduced the 1950 for marine vessels (Fig. \ref{table:sname}), 
    which is widely used in marine engineering.

    \begin{figure}[H]
        \begin{tabular}{|c|c|c|c|c|}
            \hline DOF & & $\begin{array}{c}\text { Forces and } \\
            \text { Moments }\end{array}$ & $\begin{array}{c}\text { Linear and } \\
            \text { Angular velocity }\end{array}$ & $\begin{array}{c}\text { Positions and } \\
            \text { Euler Angles }\end{array}$ \\
            \hline
            1 & Surge & $X$ & $u$ & $x$ \\
            2 & Sway & $Y$ & $v$ & $y$ \\
            3 & Heave & $Z$ & $w$ & $z$ \\
            4 & Roll & $K$ & $p$ & $\phi$ \\
            5 & Pitch & $M$ & $q$ & $\theta$ \\
            6 & Yaw & $N$ & $r$ & $\psi$ \\
            \hline
        \end{tabular}
        \caption{SNAME notation}
        \label{table:sname}
    \end{figure}

\subsection*{Frames of reference}

    In order to derive the kinematics and dynamics of the system, the calculations need to be projected to the same frame of reference.
    Sometimes it is more convinient to define several coordinate frames based on the system parameters and the measurements.
    These frames are the earth-fixed frame, which is considered to be inertial and its origin is fixed, and the body-fixed frame, which is a moving frame fixed to the vehicle, as depicted in (Fig. \ref{image:frames}). 
    \begin{figure}[H]
        \centering\includegraphics*[width=0.5\textwidth]{frames}
        \caption{The frames}
        \label{image:frames}
    \end{figure}

    The origin of the body-fixed frame usually coincides with the vehicle's center of mass,
     and the axes of this frame are chosen along the vehicle's principle axes of inertia.

    Using the SNAME notation we could define the position and the velocity vectors:    
    $$
    \begin{aligned}
    & \eta=\left[\begin{array}{llllll}
        x & y & z & \phi & \theta & \psi
        \end{array}\right]^{\top} \\
    & \nu=\left[\begin{array}{llllll}
    u & v & w & p & q & r
    \end{array}\right]^{\top} \\
    \end{aligned}
    $$
    Here $\eta$ denotes the position and orientation vector in the earth-fixed frame, 
    $\nu$ denotes the linear and angular velocity vector in the body-fixed frame.

    % Furthermore, we could define separate linear and angular components of position and velocity.\\
    % $$
    % \mathbf{p}=\left[\begin{array}{l}
    % x \\
    % y \\
    % z
    % \end{array}\right], \quad \mathbf{v}=\left[\begin{array}{l}
    % u \\
    % v \\
    % w
    % \end{array}\right], \Theta=\left[\begin{array}{l}
    % \phi \\
    % \theta \\
    % v
    % \end{array}\right], \quad \omega=\left[\begin{array}{l}
    % p \\
    % q \\
    % r
    % \end{array}\right]
    % $$
    % where $p \in \mathbb{R}^{3 x 1}$ is the linear position, $v \in \mathbb{R}^{3 x 1}$ is the linear velocity, $\theta \in \mathbb{R}^{3 x 1}$ is the angular position. $\omega \in \mathbb{R}^{3 r 1}$ is the angular velocity.
    % The force vectors acting ob the body can be seen in equation ().\\
    % $$
    % \begin{aligned}
    %     \tau=\left[\begin{array}{llllll}
    %     X & Y & Z & K & M & N
    %     \end{array}\right]^{\top} \\
    % \end{aligned}
    % $$
    % where $\tau \in \mathbb{R}^{6 x 1}$ defined in the body-fixed frame.\\
    In order to deal further with the vector transformations between these two frames 
    it is necessary to define cross product operators:\\
    $S(\lambda)$ is a skew-symmetric matrix defined such that:
    $$
    S(\lambda)=\left[\begin{array}{ccc}
        0 & -\lambda_3 & \lambda_2 \\
        \lambda_3 & 0 & -\lambda_1 \\
        -\lambda_2 & \lambda_1 & 0
    \end{array}\right]
    $$ therefore $S(u)v = u \times v$ for vectors v,u $\in \mathbb{R}^{3}$ \\
    $\bar{\times}^*$ is the $\mathbb{R}^{6}$ cross product operator defined as: \\
    
    $
    \nu\bar{\times}^*=\left[\begin{array}{ll}
        S(\omega) & 0_{3 \times 3} \\
        S(v) & S(\omega)
    \end{array}\right]
    $ where $\nu = \left[\begin{array}{l}
        v \\
        \omega
    \end{array}\right]$

\subsection*{Kinematics}

    Kinematics describes the motion of the marine vehicle without considering the forces acting upon it.
    In order to describe motion of the body, it is necessary to connect velocities in two frames with linear transformations as:
    $$
    \begin{aligned}
        & \dot{\eta}=J(\eta) \nu \\
        \text{where } & J(\eta)=\left[\begin{array}{cc}
        R(\eta) & 0_{3 \times 3} \\
        0_{3 \times 3} & T(\eta)
        \end{array}\right]
    \end{aligned}
    $$
    where the $R$ matrix is the linear transformation matrix and the $T$ matrix is the rate transformation matrix. 
    These two matrices can be expressed in terms of Euler angles, but it will cause the singularity. 
    Another way of representing orientation to avoid singularity would be quaternions.
    The quaternion kinematic equations for the marine vehicle can be expressed as follows:
    $$
    \begin{aligned}
    & \boldsymbol{R}(\boldsymbol{q})=\left[\begin{array}{ccc}
        1-2\left(q_3^2+q_4^2\right) & 2\left(q_2 q_3-q_4 q_1\right) & 2\left(q_2 q_4+q_3 q_1\right) \\
        2\left(q_2 q_3+q_4 q_1\right) & 1-2\left(q_2^2+q_4^2\right) & 2\left(q_3 q_4-q_2 q_1\right) \\
        2\left(q_2 q_4-q_3 q_1\right) & 2\left(q_3 q_4+q_2 q_1\right) & 1-2\left(q_2^2+q_3^2\right)
        \end{array}\right]\\
    & \boldsymbol{T}(\boldsymbol{q})=\frac{1}{2}\left[\begin{array}{rrr}
        q_1 & -q_4 & q_3 \\
        q_4 & q_1 & -q_2 \\
        -q_3 & q_2 & q_1 \\
        -q_2 & -q_3 & -q_4
        \end{array}\right]
    \end{aligned}
    $$
    while state quaternion $q$ is expressed in scalar-first form as 
    $q = q_1 + q_2\cdot i + q_3\cdot j + q_4\cdot k$

\subsection*{Model dynamics}

    The Newton-Euler approach is commonly used to describe the dynamics of marine vehicles.
    This approach relates the applied forces and moments to the vehicle's accelerations and angular accelerations.
    The equations of motion using the Newton-Euler approach in the body-fixed frame can be written as:
    $$
    M\dot\nu+\nu\bar{\times}^*M\nu=f
    $$
    where M represents the inertia matrix of the rigid body and 
    f represents the total force acting on it.\\
    The system of equations can be transformed into matrix form:
    $$
     \mathbf{M}_{R B} \dot{\nu}+\mathbf{C}_{R B}(\nu) \nu
    =\tau_{R B}
    $$
    where
    $M_{R B} \in \mathbb{R}^{6 x 6}$ is the rigid body mass matrix,
    $C_{R B}(\nu) \in \mathbb{R}^{6 x 6}$ is the rigid body Coriolis and centripetal forces matrix,
    $\tau_{R B} \in \mathbb{R}^{6 x 1}$ is the generalized vector of external forces and torques.
    
    To refine the dynamics model, additional terms need to be included to account for various factors. 
    These terms include added mass, which represents the inertia of the surrounding fluid, 
    the shift of the center of buoyancy due to changes in trim and heel angles, 
    and damping effects. 
    By incorporating these terms into the manipulator equation derived from the Newton-Euler approach, 
    the model becomes more accurate and reflects the true behavior of the marine vehicle.    

\subsection*{Center of Gravity and Center of Buyonancy}
    
    Due to the robust design of the marine vehicles, the center of mass(COM) placed lower than the center of byonancy(COB).
    This shift betweent centers causes torque acting against capsize (Fig. \ref{image:scheme}).\\
    \begin{figure}[H]
        \centering\includegraphics*[width=0.5\textwidth]{01_srb}
        \caption{The vehicle scheme}
        \label{image:scheme}
    \end{figure}
    If we place the origin of the body frame at byonancy center, the mass matrix can be expressed as:\\
    $$
    M_{R B}=\left[\begin{array}{cc}
        m I_{3 \times 3} & -m S\left(r_G\right) \\
        m S\left(r_G\right) & I_0
    \end{array}\right]
    $$
    where $r_G$ is the vector of the gravity center in the body frame.\\
    The same applies to the Coriolis matrix:
    $$
    C_{R B}(\nu) =\left[\begin{array}{cc}
        S(\omega) & 0_{3 \times 3} \\
        S(v) & S(\omega)
    \end{array}\right]\left[\begin{array}{cc}
        m I_{3 \times 3} & -m S\left(r_G\right) \\
        m S\left(r_G\right) & I_0
    \end{array}\right]
    $$
    These matrices can be further simplified, using an assumption that x and y shifts equal zero. 
\subsection*{Concept of added mass}

    Since the vehicle moves in a viscous enviroment, we can not neglect the inertia of the surrounging liquid.
    To compensate added mass effect, it is necessary to add two components into dynamics equation.\\
    Using SNAME(1950) notation, we can define dynamical parameters of our body as:
    $$
    Y_{\dot{u}} \triangleq \frac{\partial Y}{\partial \dot{u}}
    $$
    Consequently the added mass matrix $M_A$ and 
    the Coriolis forces matrix for added mass $C_A(\nu)$
    can be expressed as: 
    $$
    \begin{aligned}
        & M_A=\left[\begin{array}{cc}
            A_{11} & A_{12} \\
            A_{21} & A_{22}
            \end{array}\right]=-\operatorname{diag}\left\{X_{\dot{u}}, Y_{\dot{v}}, Z_{\dot{w}}, K_{\dot{p}}, M_{\dot{q}}, N_{\dot{r}}\right\} \\
        & C_A(\nu)=\left[\begin{array}{cc}
        0_{3 \times 3} & -S\left(A_{11} v+A_{12} \omega\right) \\
        -S\left(A_{11} v+A_{12} \omega\right) & -S\left(A_{21} v+A_{22} \omega)\right.
        \end{array}\right]
    \end{aligned}
    $$

\subsection*{Hydrodynamic Damping}

    Generally, the dynamics of underwater vehicles can be highly nonlinear and coupled.
    Nevetheless, during the slow non-coupled motion the damping can be approximated to linear and quadratic damping:
    $$\begin{aligned}
        & D(\nu)=-\operatorname{diag}\left\{X_u, Y_v, Z_w, K_p, M_q, N_r\right\} \\
        & \quad-\operatorname{diag}\left\{X_{u|u|}|u|, Y_{v|v|}|v|, Z_{w|w|}|w|, K_{p|p|}|p|, M_{q|q|}|q|, N_{r|r|}|r|\right\}
    \end{aligned}
    $$

\subsection*{Restoring forces}
The weight of the body is defined as: $W=m g$, 
while the buoyancy force is defined as: $B=\rho g \nabla$. 
By transforming the weight and buoyancy force to the body-fixed frame, we get:

$$
f_G\left(\eta\right)=R^{\top}\left(\eta\right)\left[\begin{array}{l}
0 \\
0 \\
W
\end{array}\right] \quad f_B\left(\eta\right)=-R^{\top}\left(\eta\right)\left[\begin{array}{l}
0 \\
0 \\
B
\end{array}\right]
$$
Therefore, overall restoring force and moment vector is:
$$
g(\eta)=-\left[\begin{array}{c}
f_G(\eta)+f_B(\eta) \\
r_G \times f_G(\eta)+r_B \times f_B(\eta)
\end{array}\right]
$$

\subsection*{Matrix representation}

The final expression for the mathematical model is:
$$
\begin{cases}
& M \dot{\nu}+C(\nu) \nu+D(\nu) \nu+g(\eta)=\tau \\
& \dot{\eta}=J(\eta) \nu
\end{cases}
$$
where
$M=M_{R B}+M_A$, $C(\nu)=C_{R B}(\nu)+C_A(\nu)$

\subsection*{Thrusters modelling}

In the general case, the thruster force and moment vector will be 
a complicated function depending on the vehicle's velocity vector $\nu$, 
voltage of the power source V 
and the control variable $u$.
This relationship can be expressed as:
$$
\tau=Bb(\nu, V, u)
$$
where $B$ is a control mapping matrix and
$b(\cdot)$ is a nonlinear vector function.

\subsection*{BlueRov modelling (will be placed in a different chapter (?))}

By the specification of the given thrusters, the dependency between control 
PWM signal and thrust is highly nonlinear
(Fig. \ref{image:thrust}).\\
\begin{figure}[H]
    \centering\includegraphics*[width=0.8\textwidth]{thrusters}
    \caption{}
    \label{image:thrust}
\end{figure}
In order to model this relation, the polynomial regression was applied on the normalized test data. 
A 5th-order approximation of the developed thrust at 16V voltage will be:
$
\tau = - 0.22 u^5
- 0.0135 u^4
+ 1.1 u^3
+ 0.172 u^2
+ 1.327 u 
+ 0.027
$\\
The inverse dependency can be determined in the same way. 
The following expression is obtained :
$
u = 0.0006 \tau^5 
- 0.0004 \tau^4 
- 0.02 \tau^3
+ 0.0006 \tau^2 
+ 0.56 \tau
-0.0334
$\\
\href{https://colab.research.google.com/drive/1XaNNENZPk88yaddOYy01vXtHd8_YwT2m?usp=sharing}{Click for Colab}