\chapter{Methodology}
\label{chap:ctrl}

\chaptermark{Forth Chapter Heading}

% viscousity
% thrusters
% control lyapunov function

\chaptermark{Third Chapter Heading}

\todo{\#TODO Reformulate intro}

    The development of a robust control system for underwater robots hinges on a 
    comprehensive understanding of system dynamics and control objectives.
    This chapter explores control design for underwater robots, focusing on the application 
    of sliding mode control to address challenges in marine environments. By integrating 
    robust control strategies, the aim is to to enhance the stability, accuracy, and 
    responsiveness of underwater robotic systems operating in dynamic and unpredictable conditions.

\section{Control objectives}
    \todo{\#TODO Change name}

    \temp{General control design}
    The main idea of the control system design is to choose such control input $u$ which 
    meets the desired performance specifications while ensuring stability.

    \temp{Underwater control design}
    Underwater robots require precise control systems to navigate and operate effectively
    in challenging marine environments. These control objectives are crucial for ensuring
    the robot's stability, accuracy, and responsiveness:
    \begin{itemize}
        \item Position and Orientation Tracking:
            The robot must accurately follow a desired trajectory, maintaining its position
            and orientation as intended.
        \item Disturbance Rejection:
            The robot should be able to withstand external disturbances, such as ocean
            currents, waves, and sensor noise, to maintain stable tracking performance.
        \item Robustness:
            The control system should be robust to uncertainties in the robot's dynamics
            and environmental conditions, ensuring reliable operation even in unpredictable
            situations.
        \item Real-Time Implementation:
            The control algorithm should be computationally efficient and able to run in
            real-time on the robot's embedded system, enabling prompt and effective
            responses to changing conditions.
    \end{itemize}

    \todo{\#TODO Connect with the next subsection}

\subsection{Model Uncertainties}

    \temp{What uncertainties? What are they?}
    The estimated model dynamics may not perfectly match the actual system behavior 
    due to imprecise parameter estimates caused by unmodeled or simplified dynamics 
    and external factors. From a control perspective, there exist two primary types 
    of modeling inaccuracies:
    \begin{enumerate}
        \item structured (or parametric) uncertainties
        \item unstructured uncertainties (or unmodeled dynamics)
    \end{enumerate}
    The first kind corresponds to inaccuracies in the model's included terms,
    while the second kind relates to inaccuracies on the
    system order.

    \temp{Notation}
    $\hat x$ indicates the approximate value of parameter $x$, while 
    the estimation error is defined as $\tilde x = \hat x - x$

    \todo{\#TODO State which parameters are unknown to us?}

    There may be some imprecision in the mass matrix of the body $M_B$, as well as 
    in the matrix of Coriolis forces $C_B$. Additionally, the exact values of damping, 
    $D$, and restoring forces, $g$, may also be imprecise.

    Due to the high dependence on current environmental conditions, the values of 
    added mass matrices, $M_A$ and $C_A$, cannot be directly calculated and will 
    be omitted in further calculations. Instead, we will incorporate a disturbance 
    term $\delta$. 

    The thruster mapping is a complex and nonlinear function that can be linearized 
    using matrix $B$.

    \todo{\#TODO Add thruster mapping approximation}

    \temp{Control affected}
    \todo{\#TODO Reformulate}\\
    As discussed earlier, modeling inaccuracies can have strong adverse effects on
    nonlinear control systems. Therefore, any practical design must address them
    explicitly. Two major and complementary approaches to dealing with model uncertainty 
    are robust and adaptive control. Futher, we are going to examine the simple approach to 
    robust control, so-called sliding control methodology.

    \temp{Conclusion}
    \todo{\#TODO Reformulate}\\
    As a result, the ultimate goal of the underwater control system is to follow the desired 
    trajectory ${\bar r}_{des}^B$ in presence of external disturbances and uncertainties.
    By measuring the difference between the actual trajectory ${ \bar r }^B$ and the 
    desired one, we obtain the tracking error:
    $
        {\tilde{\bar r}}^B = {\bar r}_{des}^B - {\bar r}^B
    $.
    The control objective can be reformulated to achieve ${\tilde{\bar r}}^B\rightarrow 0 $ 
    as $t\rightarrow \infty$.

\section{Inverse dynamics}

    \temp{Intro to Inverse Dynamics}
    Inverse dynamics is a nonlinear control technique that provides a trajectory tracking
    by calculating the required joint actuator torques to achieve a given trajectory. 
    This approach relies on exact cancelation of nonlinearities in the robot equation of motion.

    \temp{More on inverse dynamics}
    The inverse dynamics control is directly related to the solution of the inverse
    dynamics problem. By appropriately inverting the dynamic model, a control law can 
    cancel the nonlinear part of the dynamics, decouple the interactions 
    between the regulated variables, and specify the time characteristics of the decay of the 
    task errors.

    \temp{Control design}
    Recalling the system dynamics(), we can design the following control law to 
    linearize the system:
    $$
        \bar{f}^B = M {a} +
        C(\bar{v}^B)\bar{v}^B +
        D(\bar{v}^B)\bar{v}^B +
        g(\bar{r}^N)
    $$
    where ${a}$ is outer-loop control to be designed further. \\
    Substitution to the dynamics yields the double integrator system:
    $$\dot{\bar{v}}^B = a$$
    The outer loop control $a$ is usually defined as a proportional-derivative (PD) controller:
    $$a = \dot{\bar{v}}^B_{des} - K_p \tilde{{\bar{r}}}^B - K_d \tilde{{\bar{v}}}^B$$

    \temp{Introduce uncertainties}
    It's crucial to note that in order to calculate the necessary control inputs to achieve 
    a desired trajectory using the inverse dynamics technique, an accurate model of the 
    system dynamics is required. However, when dealing with nonlinear systems that involve 
    uncertainties and disturbances, this approach may not be effective. The model used for 
    inverse dynamics may not fully capture the intricate behavior of the system under 
    varying conditions, leading to suboptimal or unstable control inputs. 
    
    As a result, the inverse dynamics technique may not be the best option for effectively 
    controlling underwater systems.

    \todo{\#TODO Add more theory}
    
    \todo{\#TODO Add error dynamics and analysis}

\section{Sliding Mode}

    \temp{Intro to SMC}
    As discussed in the previous chapter, there are several controller
    designs available. However, the sliding mode approach suggested by \todo{(Spong - ?)}
    is highly regarded as the most sophisticated and frequently implemented one.

    \temp{SMC deffinition}
    Sliding mode control (SMC) is a nonlinear control method that guarantees
    robust control of systems with uncertainties and disturbances.
    This technique involves developing a sliding surface within the state
    space and directing the system's trajectory to slide along this surface (Fig. \ref{image:sliding_mode}).

    \begin{figure}[H]
        \centering\includegraphics*[width=0.5\textwidth]{sliding_mode}
        \caption{The general sliding mode scheme}
        \label{image:sliding_mode}
    \end{figure}

    \temp{Advantages to use SMC}
    Compared to other nonlinear control methods, SMC is a relatively straightforward
    solution to implement with a basic understanding of system dynamics and sliding
    surface design. SMC provides a fast transient response due to the sliding dynamics, which makes 
    it possible to track desired references or trajectories quickly. Additionally, 
    the sliding surface ensures robustness to uncertainties and disturbances by 
    making the system behavior insensitive to these factors. In summary, SMC is a 
    simple and effective solution for controlling nonlinear systems.

\subsection{Sliding Surface Design}

    \temp{Sliding surface}
    In sliding mode control, a sliding surface is a hyperplane in
    the state space that defines the desired system behavior.
    The control objective is to force the system's trajectory to slide
    long this surface. Once the system's trajectory reaches the sliding surface,
    it will remain on the surface as long as the control law is applied.

    \begin{figure}[H]
        \centering\includegraphics*[width=0.9\textwidth]{sliding_phases}
        \caption{The phases of sliding mode}
        \label{image:sliding_phases}
    \end{figure}

    \temp{Properties of SMC}
    The design of the sliding surface is critical for the performance of the SMC
    system. The sliding surface should be:
    \begin{itemize}
        \item Reachable: The system's trajectory should be able to reach the sliding
            surface in a finite amount of time.
        \item Invariant: Once the system's trajectory reaches the sliding surface, it
            should remain on the surface for all future time.
        \item Attractive: The control law should attract the system's trajectory to the
            sliding surface and keep it there.
    \end{itemize}

    \temp{Invariant sets}
    In order to satisfy the conditions above, the sliding surface is designed to be an invariant set.
    Invariant sets are sets of states in the state space that, once entered, cannot be exited under 
    the action of the control law.

    \temp{General equation}
    Let us define time-varying surface $\mathcal{S}$ in the the state space $\mathbb{R}^n$
    given by scalar equation $s(\bar{r}^B, t)$:
    $$
        s(\bar{r}^B, t) = (\frac{d}{dt} + \lambda)^{n-1}\tilde{\bar{r}}^B
    $$
    where $n$ is the order of the system and \todo{$\lambda$ is a positive scalar?}

    \temp{Sliding condition}
    Let as define Lyapunov candidate $V = s^2$ as the squared distance to the surface.
    In order to ensure convergence along all system trajectories one may
    formulate the following sliding condition:
    $$
    \frac{dV}{dt} < -\eta \sqrt{V} \text{\quad or \quad}
    \frac{1}{2}\frac{d}{dt}\|s\|^2 = s^T\dot{s} < \eta \|s\|
    $$
    where $\eta>0$ defines the rate of convergence to the sliding surface.

    \todo{\#TODO Add more info}

    \temp{Convergence proof}
    Satisfying condition or sliding condition, makes the surface an invariant set
    and implies convergence to $\tilde{r}^B$, since:
    $$
        s = (\frac{d}{dt} + \lambda)^{n-1}\tilde r^B = 0
    $$

    \todo{\#TODO Add more info about differecntial equation. Add images.}

    \temp{New control objective}
    Applying such transformation yields a new representation of the tracking performance:
    $$
        s \rightarrow 0 \Rightarrow \tilde{r}^B \rightarrow 0
    $$
    Meaning, that the problem of tracking $r^B$ is equivalent to remaining on
    the sliding surface. Thus the problem of tracking the $n$-dimensional vector $r^B$
    can in effect be replaced by a first order stabilization problem in $s$.

\subsection{Control Law Design}

\begin{figure}[H]
    \centering\includegraphics*[width=0.8\textwidth]{control_scheme}
    \caption{The sliding mode control scheme}
    \label{image:control_scheme}
\end{figure}

    \temp{Control law}
    The controller comprises two distinct components: nominal control $a_n$, 
    and an additional robustifying part $a_s$ (Fig. \ref{image:control_scheme}): 
    $$
        a = a_n + a_s
    $$
    The nominal control $a_n$ aims to compensate for the system's dynamics, 
    while the robustifying component $a_s$ is designed to enhance the 
    controller's stability and performance by providing additional corrective 
    action to counteract uncertainties and disturbances.

    \temp{System dynamics with uncertainties}
    By taking into account the impact of both dynamic approximations 
    and disturbances, we can obtain the following dynamics of the 
    system:
    $$
    M \dot{\bar{v}}^B + C(\bar{v}^B) \bar{v}^B+D(\bar{v}^B) \bar{v}^B+g(\bar{r}^N) + \delta = BKu
    $$

    \temp{Inverse dynamics}
    \todo{\#TODO Check formulas}\\
    Using the inverse dynamics approach, we can apply the outer loop 
    controller to partially linearize the system with model estimates:
    $$
    \hat u = \hat{B}^{-1}(\hat{M}a + \hat{C}(v^B)v^B + \hat{D}(v^B)v^B + \hat{g}(r^B))
    $$
    Substitution to the dynamics yields the equation:
    $$
        \dot{\bar{v}}^B = F(\bar{r}^N, \bar{v}^B) + Ka
    $$
    with $F = 
    M^{-1}(B\hat{B}^{-1}\tilde{f} + (B\hat{B}^{-1}-I)f + \delta)$, 
    $f = 
    C(\bar{v}^B) \bar{v}^B + D(\bar{v}^B) \bar{v}^B + g(\bar{r}^N)$
    and $K = M^{-1}B\hat{B}^{-1}\hat{M}$

    \temp{Sliding condition}
    The time derivative of $s$ is connected to dynamics as follows:
    $$
    \dot{s} = \dot{\tilde{\bar{v}}}^B + \lambda \tilde{\bar{v}}^B = 
    a_n - \dot{\bar{v}}^B = a_n - F - K(a_n + a_s) = w - Ka_s 
    $$
    with $w = (I - K)a_n - F $

    Substitution to sliding condition yields:
    $$
    s^Tw- s^TKa_s  \leq \|s\|\|w\| - s^TKa_s  \leq - \eta \|s\|
    $$
    Thus we can choose the $a_s$ as:
    $$
        a_s = \frac{k}{\sigma_{max}}\hat{M}^{-1}\frac{s}{\|s\|} = 
        \rho \frac{s}{\|s\|} 
    $$
    \todo{\#TODO Describe idea clearly}
    \begin{figure}[H]
        \centering\includegraphics*[width=0.5\textwidth]{matrix_boundary}
        \caption{The boundaries for the matrix.}
        \label{image:matrix_boundary}
    \end{figure}
    where $\sigma_{max}$ is maximal singuilar value of \todo{$M^{-1}$}
    which provide:
    $$
        \|s\|\|w\| - s^TKa_s \leq \|s\|\|w\| \todo{+} 
        \frac{k}{\sigma^2_{max}\|s\|}s^TM^{-1}s \leq
        \|s\|\|w\| \todo{+} k \|s\| < - \eta \|s\|
    $$
    Setting gain $k$ accordingly to:
    $$
        k > \|w\| + \eta
    $$
    will satisfy sliding conditions.

    The final expression for sliding control:
    $$    
    a_s = 
    \begin{cases}
    \rho \frac{s}{\|s\|}, \quad \|s\| >\epsilon\\
    0, \quad \|s\| = 0 
    \end{cases}
    $$

    \temp{Nominal torque}
    The nominal control $a_n$
    designed in a way:
    $$
        a_n = - K_0\tilde{v}^B - K_1\tilde{r}^B
    $$

    \temp{Solve chattering problem}
    In order to reduce chattering, the controller above is effectively smoothed using
    the boundary layer:
    $$
    a_s = 
    \begin{cases}
    \rho \frac{s}{\|s\|}, \quad \|s\| >\epsilon\\
    \rho \frac{s}{\epsilon}, \quad \|s\| \leq\epsilon
    \end{cases}
    $$
    where $\epsilon$ is the boundary layer thickness.

\begin{figure}[H]
    \centering\includegraphics*[width=0.6\textwidth]{boundary}
    \caption{The sliding mode scheme with boundary layer}
    \label{image:boundary}
\end{figure}

    \temp{Final law}
    The resulting controller is then given as follows:
    \begin{align*} 
        &\hat u = \hat{B}^{-1}(\hat{M}a + \hat{C}(v^B)v^B + \hat{D}(v^B)v^B + \hat{g}(r^B)) \\
        &a = a_n + a_s \\
        &a_n = - K_0\tilde{v}^B - K_1\tilde{r}^B \\
        &a_s = 
        \begin{cases}
        \rho \frac{s}{\|s\|}, \quad \|s\| >\epsilon\\
        \rho \frac{s}{\epsilon}, \quad \|s\| \leq\epsilon
        \end{cases} \\
        &s = \tilde{\bar{v}}^B + \lambda \tilde{\bar{r}}^B
    \end{align*}

\section{Optimization control}

\section{Stability analysis}
% % lyapunov candidate
% In order to prove global stability of the system, let the Lyapunov candicate will be:
% $$
%     V = q^TPq
% $$
% Let us take the time derivative, we will get
% $$
%     \dot V = ...
% $$

% % LaSalle theorem
% However, we need to discover if $\dot V = 0$ even while $e \neq 0$.
% According to LaSalle theorem ...

\section{Summary}

\todo{\#TODO Summarize the main findings of the chapter.}

The methodology of robust control via sliding mode can be formulated in following steps:
\begin{itemize}
    \item Define the sliding surface $s(y, t)$
    \item Derive nominal control $\hat{u}$ (our best guess) that may achive $\dot{s} = 0$
    \item Modify control law by discontinuse term that will bring system to sliding mode
    by satisfying the sliding condition.
\end{itemize}

% \todo{Discuss the limitations and potential extensions of the SMC
%     controller.}


