\chapter{Methodology}
\label{chap:ctrl}

\chaptermark{Forth Chapter Heading}

% robust control
% viscousity
% thrusters
% control lyapunov function

\chaptermark{Third Chapter Heading}

% Chapter intro

\section{Control design}

\todo{Introduce the concept of sliding mode control (SMC) and 
its advantages for controlling underwater robots.}


As discussed in the previous chapter, there are several controller 
designs available. However, the sliding mode approach suggested by \temp{(Spong - ?)}
is highly regarded as the most sophisticated and frequently implemented one. 

Sliding mode control (SMC) is a nonlinear control method that guarantees 
robust control of systems with uncertainties and disturbances. 
This technique involves developing a sliding surface within the state 
space and directing the system's trajectory to slide along this surface (Fig. \ref{image:sliding_mode}).

\begin{figure}[H]
    \centering\includegraphics*[width=0.5\textwidth]{sliding_mode}
    \caption{The sliding mode scheme}
    \label{image:sliding_mode}
\end{figure}


Underwater robots face a range of challenges, including ocean currents, waves, 
and sensor noise. Fortunately, SMC provides a reliable and 
stable solution to these issues, making it an excellent choice for underwater 
robotics.

Additionally, SMC is known for its ability to respond quickly and accurately track desired 
trajectories, even in the presence of disturbances. This is particularly important 
for underwater robots that must navigate, dock, and manipulate objects.

Compared to other nonlinear control methods, SMC is a relatively straightforward 
solution to implement. With a basic understanding of system dynamics and sliding 
surface design.

\temp{the Leitmann approach - ?}

\subsection{Problem Formulation}

\todo{Define the underwater robot system dynamics.\\}
The final system dynamics is represented as:
$$
M \dot{\bar{v}}^B + C(\bar{v}^B)\bar{v}^B + D(\bar{v}^B)\bar{v}^B + g(\bar{r}^N)= \bar{f}^B
$$

\todo{Define control objectives\\}
Underwater robots require precise control systems to navigate and operate effectively
in challenging marine environments. These control objectives are crucial for ensuring
the robot's stability, accuracy, and responsiveness:
\begin{itemize}
    \item Position and Orientation Tracking: 
    The robot must accurately follow a desired trajectory, maintaining its position 
    and orientation as intended.
    \item Disturbance Rejection: 
    The robot should be able to withstand external disturbances, such as ocean 
    currents, waves, and sensor noise, to maintain stable tracking performance.
    \item Robustness: 
    The control system should be robust to uncertainties in the robot's dynamics 
    and environmental conditions, ensuring reliable operation even in unpredictable 
    situations.
    \item Real-Time Implementation: 
    The control algorithm should be computationally efficient and able to run in 
    real-time on the robot's embedded system, enabling prompt and effective 
    responses to changing conditions.
\end{itemize}

\temp{Derive the state-space representation of the system - ?}

\todo{Identify the uncertainties and disturbances affecting the system.\\}
The dynamics parameter estimates may be imprecise due to unmodeled dynamics and external 
factors. This means the estimated values might not perfectly match the actual system behavior.\\
$\hat x$ indicates the approximate value of parameter $x$, while 
the estimation error is defined as $\tilde x = \hat x - x$.

\temp{State which parameters are unknown to us - ?}

\subsection{Sliding Surface Design}
\todo{Explain the concept of a sliding surface and its role in SMC.\\}
In sliding mode control (SMC), a sliding surface is a hyperplane in 
the state space that defines the desired system behavior. In SMC, the 
sliding surface is designed to be an invariant set.

Invariant sets are sets of states in the state space that, once entered,
cannot be exited under the action of the control law.\\
For an invariant set $\mathbb{S}$ the following is valid:\\
if $x(t_0) \in \mathbb{S}$ 
then $x(t) \in \mathbb{S}$ for $\forall t > t_0 $\\
\temp{
Sliding surface is invariant set if it satisfies the sliding condition:
$$
\frac{1}{2}\frac{d}{dt}s^2 = - \eta |s|
$$}
The control objective is to force the system's trajectory to slide 
long this surface. Once the system's trajectory reaches the sliding surface, 
it will remain on the surface as long as the control law is applied.

The sliding surface provides robustness to uncertainties and 
disturbances by ensuring that the system's behavior is insensitive 
to these factors. This is because the control law is designed to counteract any 
disturbances or uncertainties that would push the trajectory off 
the surface. As long as the system's trajectory remains on the 
sliding surface, the control system will maintain stability and 
performance.

Therefore, the initial tracking problem $\bar{r}^B = \bar{r}_{des}^B$
 is equivalent to remaining on the surface $\mathbb{S}$ or $s(t) = 0$ for $\forall t > t_0$.

\todo{Derive the sliding surface for the underwater robot system.\\}
General sliding surface for the system is:
$$
s = (\frac{d}{dt} + \lambda)^{n-1}\tilde r^B = \tilde v^B + \lambda \tilde r^B
$$
where $n$ is the order of the system.

\todo{Analyze the properties of the sliding surface, 
such as reachability and invariance\\}
The design of the sliding surface is critical for the performance of the SMC system. The sliding surface should be:
\begin{itemize}
    \item Reachable: The system's trajectory should be able to reach the sliding surface in a finite amount of time.
    \item Invariant: Once the system's trajectory reaches the sliding surface, it should remain on the surface for all future time.
    \item Attractive: The control law should attract the system's trajectory to the sliding surface and keep it there.
\end{itemize}
\temp{Global invariant set theorem - ?}

\subsection{Control Law Design}

\todo{Derive the SMC control law for the underwater robot system.\\}
In order to find a minimum of $s$, which corresponds to the minimal tracking error,
let us take a time derivative:
$$
\dot{s} = \dot{\tilde v}^B + \lambda \dot {\tilde r}^B = 
    \dot{\hat v}^B - \dot{v}^B + \lambda v^B = 0
$$
Substituting the system dynamics, it would give us the following equation:
$$
\dot{s} = - M^{-1}(C(v^B)v^B + 
    D(v^B)v^B + g(r^N) - f^B) - \dot{v}^B + \lambda {v}^B = 0 
$$
Note that inertia matrix $M$ is always invertible by the construction.\\
The final expression for control force is:
$$
f^B = - Ma + C(v^B)v^B + D(v^B)v^B + g(r^N)
$$
where $a = \dot{v}^B + \lambda v^B$ is outer-loop control.

\temp{
In order to form a linear closed-loop system, we will choose control input according to: 
$$
\bar{f}^B = \hat M {a} +
            \hat C(\bar{v}^B)\bar{v}^B +
            \hat D(\bar{v}^B)\bar{v}^B + 
            \hat g(\bar{r}^N)
$$
where ${a}$ is outer-loop control to be designed further. 
}

% double integrator
\temp{
If we substitute () into (), we will get double integrator system in a form:
$$
\dot{\bar{v}}^B = a + \eta (\bar{r}^N, \dot{\bar{v}}^B, a)
$$
where the uncertainty function $\eta$ is defined as $$\eta (\bar{r}^N, \dot{\bar{v}}^B, a) = 
                                        M^{-1}(\tilde M {a} +
                                        \tilde C(\bar{v}^B)\bar{v}^B +
                                        \tilde D(\bar{v}^B)\bar{v}^B + 
                                        \tilde g(\bar{r}^N))$$
}
\temp{
% outer-loop control
In order to ensure global stability of the system, the outer loop control $a$ designed in a way:
$$
a = a_{des}(t) - K_0v - K_1 r - \delta \alpha
$$
where $\delta \alpha = \begin{cases}
    \rho\frac{e}{|e|} , & \text{if } |e| > 0 \\
    0, & \text{if } |e| = 0 
\end{cases}$
}

\temp{Define the error e - ?}

\temp{
% outer-loop control
In order to reduce chattering, the boundary layer is introduced as:\\
$\delta \alpha = \begin{cases}
    \rho\frac{e}{|e|} , & \text{if } |e| \geq \epsilon \\
    \frac{\rho}{\epsilon}e, & \text{if } |e| < \epsilon 
\end{cases}$\\
where $\epsilon$ is the boundary thickness.
}

\begin{figure}[H]
    \centering\includegraphics*[width=0.8\textwidth]{boundary}
    \caption{The sliding mode scheme with boundary layer}
    \label{image:boundary}
\end{figure}

\todo{Analyze the stability of the closed-loop system under SMC.\\}
% lyapunov candidate
In order to prove global stability of the system, let the Lyapunov candicate will be:
$$
V = q^TPq
$$
Let us take the time derivative, we will get
$$
\dot V = ...
$$

% LaSalle theorem
However, we need to discover if $\dot V = 0$ even while $e \neq 0$. 
According to LaSalle theorem ...

\subsection{Summary}

\todo{Summarize the main findings of the chapter.}

\todo{Discuss the limitations and potential extensions of the SMC 
controller.}


