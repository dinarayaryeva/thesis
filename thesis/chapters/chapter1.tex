\chapter{Introduction}
\label{chap:intro}

\section{Background}
Robotic systems, particularly underwater vehicles such as Remotely Operated Vehicles 
(ROVs), have garnered significant attention in recent years due to their wide-ranging 
applications in industrial, scientific, and military domains. ROVs are employed for 
diverse purposes, including underwater exploration, pipeline inspection, and 
environmental monitoring. However, controlling these vehicles is challenging due to 
the highly nonlinear and unpredictable nature of the underwater environment. 
Therefore, effective control strategies are essential to ensure the stability and 
performance of ROVs in such conditions.

Control strategies for ROVs have evolved, with robust 
control methods like Sliding Mode Control (SMC) gaining prominence due to their 
effectiveness in handling system nonlinearities and uncertainties.
Despite these advancements, existing control methods for ROVs face several 
challenges. Traditional SMC techniques, while robust, often lead to a phenomenon 
known as "chattering," which can cause wear and tear on mechanical components and 
reduce the overall efficiency of the control system. Additionally, many control 
algorithms struggle to adapt to real-time changes in the underwater environment, 
such as varying current strengths and unforeseen obstacles. These limitations 
underscore the need for improved control strategies that can offer both robustness 
and adaptability.

\section{Research Objective}

The primary objective of this study is to enhance the control of remotely 
operated vehicles (ROVs) by developing an improved Sliding Mode Control 
(SMC) scheme that leverages convex optimization techniques. This thesis aims to 
address the existing issues of chattering and adaptability by dynamically 
adjusting control parameters through convex optimization.

\textbf{The research question}: 
How can sliding mode control for robotic systems be improved using convex 
optimization to enhance stability and performance while mitigating 
the chattering effect?

The research objectives are as follows:
\begin{enumerate}
    \item Develop a comprehensive mathematical model for ROVs that captures the essential dynamics and uncertainties
    \item Design a sliding mode control scheme enhanced by convex optimization
    \item Evaluate the performance of the proposed control scheme through simulation
\end{enumerate}

\section{Structure of the Thesis}
    The remainder of the work is structured as follows:
    Chapter 2 provides an overview of existing research on ROVs, 
    focusing on their modeling and control strategies.
    In Chapter 3, the development of the 
    mathematical model for the ROV is explored.
    Further, Chapter 4 describes the design of the 
    enhanced sliding mode control scheme.
    Chapter 5 presents the results of the simulation and experimental 
    validation, and discusses the findings.
    Finally, Chapter 6 summarizes the research contributions and suggests directions for future work.
